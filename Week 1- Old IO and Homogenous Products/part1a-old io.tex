\documentclass[aspectratio=169,11pt]{beamer}
\usepackage{teaching_slides}


\title [``Old'' IO]{Before there was ``New'' Empirical IO}
\author{C.Conlon}
\institute{Grad IO }
\date{Fall 2025}

\begin{document}

\begin{frame}
\titlepage
\end{frame}

\begin{frame}{Early Stuff}
This lecture is a bit different from all of the others
\begin{itemize}
\item Focus is primarily on theory rather than empirics
\item History of approaches (some of which have fallen out of fashion).
\item Should be familiar to most of you
\begin{itemize}
\item Brush up on first few chapters of Tirole (1988) (somewhat dated) but still the best reference for oligopoly theory.
\item Vives (2000) is a more modern (and focused) review of oligopoly theory.
\item I assume familiarity with an undergrad text like Carlton and Perloff (1999), Cabral (2000) or Shy (1996).
\end{itemize}
\end{itemize}
\end{frame}

\begin{frame}{History of IO: Part 0}
Early 20th century Agricultural Economics
\begin{itemize}
\item How can we estimate supply and demand from data?
\item Mostly homogenous agricultural products.
\item Early discussion of simultaneity/endogeneity econometrics
\end{itemize}
Complaint: everything still perfectly competitive.
\end{frame}

\begin{frame}{History of IO: Part I}
Structure-Conduct-Performance (1940-1960) Harvard
\begin{itemize}
\item Associated with the work of Joe Bain.
\item Structure (number of firms, market shares, etc.). Barriers to entry are key.
\item Structure $\rightarrow$ conduct (ie: how firms behave)
\item Conduct $\rightarrow$ performance (ie: prices, markups, efficiency)
\item Use accounting data to get performance (profits, price-cost margins, etc.)
\item OLS regression across industries to see whether profits are higher in more concentrated industries.
\item Empirics were somewhat atheoretic.
\end{itemize}
Complaint: the direction of causality is assumed. (Don't profits determine number of entrants too?).
\end{frame}

\begin{frame}{History of IO: Part II}
Chicago School (1960-1980)
\begin{itemize}
\item Most associated with the work of George Stigler and later Robert Bork ``Antitrust Paradox''
\item Monopoly is more often alleged than confirmed
\item Even when monopoly does exist -often only temporary (did MSFT take over the world?)
\item Entry and threat of entry is crucial.
\item Emphasis on price theory (markets work) and better econometrics
\item Still quite persuasive for practice of antitrust (judges and lawyers).
\end{itemize}
\end{frame}


\begin{frame}{History of IO: Part III}
Game Theory (1980-1990s)
\begin{itemize}
\item For most of the 1980s, IO was dominated by game theorists.
\item Strategic decision making, Nash Equilibrium
\item Lots of intuitive (and sometimes counter-intuitive) clean theoretical models
\item Hard to know which model is the right model for the industry we are looking at.
\end{itemize}
\end{frame}


\begin{frame}{History of IO: Part IV}
The not so ``new'' anymore empirical IO (NEIO) (1989-)
\begin{itemize}
\item Back to one industry at a time.
\item Careful game-theoretical model of industry behavior
\item Joined with modern econometrics, data, and computation.
\end{itemize}
\end{frame}


\begin{frame}{The Monopoly Problem}
Start with a quantity-setting monopolist facing a known inverse demand curve $P(Q)$ and costs $C(Q)$.
\begin{eqnarray*}
\pi(Q) = P(Q) \cdot Q - C(Q) -  F
\end{eqnarray*}
Take the FOC and derive the \alert{Lerner Index}:
\begin{eqnarray*}
\pi'(Q) &=& 0  \\
 \underbrace{\overbrace{P'(Q) \cdot Q}^{\text{\alert{monopoly distortion}}}+ P(Q)}_{MR}  &=&  \underbrace{C'(Q)}_{MC} \\
  \frac{P(Q) - C'(Q)}{P(Q)} &=& - \frac{P'(Q) \cdot Q}{P(Q)} = \frac{1}{|\epsilon_d|}
\end{eqnarray*}
\begin{itemize}
\item This is known as the \textit{Lerner (1934) Index} or \alert{economic markup}.
\end{itemize}
\end{frame}

\begin{frame}{The Monopoly Problem}
We could have rewritten it as 
\begin{eqnarray*}
P \left( 1+\frac{1}{\epsilon_d} \right) = MC
\end{eqnarray*}
\begin{itemize}
\item This is helpful because it shows us the important result that the monopolist never produces in the inelastic portion of the demand curve. $\epsilon_d \in (-1,0]$.
\item Why? MR is negative! Reduce Quantity!
\item Often data report: $\frac{P}{MC}= \mu$. But we usually work with the Lerner formula in IO.
\item For the monopolist firm level elasticity $\epsilon_d$ is the same as $\epsilon_D$ the market elasticity.
\end{itemize}
\end{frame}


\begin{frame}{Cournot Model (1838) / Nash in Quantities}
\small
\begin{itemize}
\item Assume constant marginal cost $c_i = c$ and $n$ equal sized firms to make life easy. 
\item We let $Q=\sum_{i=1}^N q_i$ the total output of the industry.
\end{itemize}
We consider profits and FOC's:
\begin{eqnarray*}
\pi_i(q_i) &=& (P(Q) - C_i'(q_i) ) \cdot q_i\\
\frac{\partial \pi_i(q_i)}{\partial q_i} &=& (P(Q) - C_i'(q_i))  +  q_i  \cdot P'(Q) \cdot \frac{\partial Q}{\partial q_i}  = 0
\end{eqnarray*}
Cournot competition implies that $\frac{\partial Q}{\partial q_i} = 1$ and $\frac{\partial q_j}{\partial q_i} = 0$ for $i \neq j$ (this is because it is a simultaneous move game).
\begin{eqnarray*}
P(Q) + P'(Q) \cdot q_i = \underbrace{P(Q) + \overbrace{P'(Q) \cdot \frac{Q}{n}}}^{\text{\alert{Cournot Distortion}}}_{MR} =  mc
\end{eqnarray*}
\end{frame}

\begin{frame}{Cournot Model (1838) / Nash in Quantities}
Rearrange to form the Lerner Index:
\begin{eqnarray*}
\frac{P-mc}{P} = - \frac{1}{n} \frac{Q}{P} P'(Q) = - \frac{1}{n \cdot \epsilon_D}
\end{eqnarray*}
Some notes
\begin{itemize}
\item In general market demand is much less elastic than firm level demand.
\item When things are symmetric then we can relate aggregate to firm level elasticity: $\epsilon_d = n \cdot \epsilon_D$.
\item For beer market demand $\epsilon_D \approx -0.8$. If $n=5$ then a typical firm faces an elasticity of $-4.0$.
\item We can back out implied markups pretty easily: $P  = \frac{MC}{1-(1/\epsilon_d)}  = \frac{4}{3} MC$.
\item Market demand can be at inelastic part of curve -- but firm level demand cannot.
\end{itemize}
\end{frame}


\begin{frame}{Betrand Paradox (1883)/ Nash in Prices}
Briefly contrast with Bertrand
\begin{itemize}
\item Two firms with symmetric marginal costs $c_i=c$.
\item Nash in prices means that $p = c$.
\item This is not very interesting or helpful.  Also firms make profits!
\item Solutions

\begin{itemize}
\item Add capacity constraints (starts to behave like Cournot again (Kreps Scheinkman)).
\item Add other frictions (search costs?)
\item Add product differentiation (mostly we focus on this).
\end{itemize}
\end{itemize}
\end{frame}


\begin{frame}{Asymmetric Cournot and HHI}
\begin{itemize}
\item Symmetry doesn't seem like a particularly realistic assumption.
\item We can extend this to the asymmetric case pretty easily by modifying the \alert{Cournot distortion}: $q_i \cdot P'(Q) \cdot \frac{\partial Q}{\partial q_i}$.
\item Instead we have that $\frac{q_i}{Q} \cdot \frac{\partial Q}{\partial q_i} = \frac{q_i}{\sum_{j=1}^n q_j} \equiv s_i$ or \alert{market share}.
\item Obviously this nests symmetric case where $q_i = \frac{Q}{n}$ or $s_i = \frac{1}{n}$.
\item The Cournot markup / Lerner Index is just
\begin{eqnarray*}
\frac{P-mc_i}{P} = \frac{s_i}{|\epsilon_D|}
\end{eqnarray*}
\item Cournot: markups are proportional to market-share.
\item Nests perfect competition $n \rightarrow \infty$ or $s_i \rightarrow 0$.
\item Semi-joke: IO economists say something is \alert{intuitive} if it follows Cournot predictions.
\end{itemize}
\end{frame}


\begin{frame}{Asymmetric Cournot and HHI}
Now consider the market share weighted Lerner index:
\begin{eqnarray*}
HHI = \sum_{i=1}^N \frac{P - mc_i}{P} s_i = \sum_{i=1}^n \frac{s_i^2}{\epsilon_D}
\end{eqnarray*}
\begin{itemize}
\item For $\epsilon_D =1$, this is known as the \alert{Hirschman-Herfindal Index}.
\item This gives us a measure of \alert{market concentration} that varies from 0 to 10,000 (units of $s_i$ are in percentages).
\item DOJ/FTC describe markets as:
\begin{itemize}
\item Highly Concentrated: $HHI \geq 2500$.
\item Moderately Concentrated: $HHI \in [1500,2500]$. $\Delta HHI \geq 250$ merits scrutiny.
\item Un-Concentrated: $HHI \leq 1500$.
\end{itemize}
\end{itemize}
\end{frame}

\begin{frame}{Asymmetric Cournot and HHI}
\begin{itemize}
\item Can also work backwards form HHI to get effective ``number of firms''.
\item Here HHI is in units of $[0,1]$ instead of $[0,10000]$.
\begin{eqnarray*}
HHI = \sum_{i=1}^N s_i^2 = \frac{1}{n^*} \rightarrow n^{*} = \frac{1}{HHI}.
\end{eqnarray*}
\item ex. Four firms with shares $40\%, 30\%, 15\%, 15\%$. So the $HHI =.295$. Thus $n^{*} = 1/.295 = 3.39$ and $\epsilon_d = \epsilon_D \cdot 3.39$.
\item Alternatively (under Cournot only!) can write:
\begin{eqnarray*}
\frac{P-MC}{P} = \frac{HHI}{\epsilon_D}
\end{eqnarray*}
\end{itemize}
\end{frame}

\begin{frame}{HHI and Welfare}
Under Cournot (and only Cournot) with constant MC, we can relate $HHI$ to particular measures of welfare:
\begin{itemize}
\item Cowling Waterson (1976) relate $HHI$ to producer share of revenue:
\begin{align*}
HHI =  \epsilon_d \cdot \frac{PS}{R}
\end{align*}
\item Spiegel (2020) relates $HHI$ to producer share of surplus:
\begin{align*}
HHI &=  \frac{1}{\epsilon_d \left(Q^{*}\right)} \cdot \frac{PS}{CS}\\
\frac{C S}{TS} &=\frac{1}{1+\epsilon_d \left(Q^{*}\right) \cdot HHI}
\end{align*}
\end{itemize}
\end{frame}

\begin{frame}{Alternatives to HHI}
\begin{itemize}
\item Another alternative is the $k$ firm concentration ratio $CR_k = \sum_{i=1}^N s_i$.
\item This can be useful as an additional descriptive statistic.
\item It shows up in some older work
\item $4$ firms is a popular measure.
\end{itemize}
\end{frame}


\begin{frame}{Complaints about HHI}
\begin{itemize}
\item HHI only relates to market power under the Cournot assumptions
\begin{itemize}
\item Holding competitor's output responses fixed so that $\frac{\partial Q}{\partial q_i} =1$.
\item Competition is about setting quantity rather than price: strong restrictions on cross-price elasticities.
\item Is quantity (instead of price) the relevant strategic variable? (Sometimes...).
\end{itemize}
\item Assumes that products are \alert{homogenous} so that all firms/products are equally good competitors.
\item More concentrated markets have higher markups, but not always lower welfare (allocating production from low to high cost firms might improve welfare).
\end{itemize}
Also, how do we \alert{define markets} in the first place?
\end{frame}



\end{document}













































