\def\beamerclassoptions{[xcolor=pdftex,dvipsnames,table,mathserif,aspectratio=169]}
\documentclass[10pt,aspectratio=169]{beamer}

% silence some Metropolis warnings
\usepackage{silence}
\WarningFilter{beamerthememetropolis}{You need to compile with XeLaTeX or LuaLaTeX}
\WarningFilter{latexfont}{Font shape}
\WarningFilter{latexfont}{Some font}

% define custom colors
\usepackage{xcolor}
\definecolor{dark gray}{HTML}{444444}
\definecolor{light gray}{HTML}{777777}
\definecolor{dark red}{HTML}{BB0000}
\definecolor{dark green}{HTML}{00BB00}
\definecolor{RoyalBlue}{cmyk}{1, 0.50, 0, 0}

% configure metropolis
\usetheme[numbering=fraction]{metropolis}
\setbeamercolor{background canvas}{bg=white}
\setbeamercolor{frametitle}{bg=dark gray}
\setbeamercolor{alerted text}{fg=dark red}
\setbeamercolor{item projected}{bg=dark red}
\setbeamercolor{local structure}{fg=dark red}
\setbeamersize{text margin left=0.5cm,text margin right=0.5cm}
\setbeamercovered{transparent=10}

% use thicker lines
\makeatletter
\setlength{\metropolis@titleseparator@linewidth}{1pt}
\setlength{\metropolis@progressonsectionpage@linewidth}{1pt}
\makeatother

% custom bullet points
\setbeamertemplate{itemize item}{\color{dark red}$\blacktriangleright$}
\setbeamertemplate{itemize subitem}{\color{dark red}$\blacktriangleright$}
\setbeamertemplate{itemize subsubitem}{\color{dark red}$\blacktriangleright$}
\newcommand{\custombullet}{{\color{dark red}$\blacktriangleright$}\hspace{0.5em}}


% imports
\usepackage[english]{babel}
\usepackage[utf8]{inputenc}
\usepackage{amsthm}
\usepackage{amssymb}
\usepackage{amsmath}
\usepackage{amsfonts}
\usepackage{mathtools}
\usepackage{mathabx}
\usepackage{stmaryrd}
\usepackage{graphicx}
\usepackage{hyperref}
\usepackage{xfrac}
\usepackage{appendixnumberbeamer}
\usepackage{tabularx}

% check and x marks
\usepackage{pifont}
\newcommand{\cmark}{{\color{dark green}\ding{51}}\hspace{0.3em}}
\newcommand{\xmark}{{\color{dark red}\ding{55}}\hspace{0.5em}}


% use classic font for math
\usepackage[T1]{fontenc} % Needed for Type1 Concrete \usepackage{concmath}
\usefonttheme{serif}
\usefonttheme{professionalfonts}
\usepackage{concmath}
\setbeamerfont{equation}{size=\tiny}



% diagrams
\usepackage{tikz}
\usetikzlibrary{decorations.pathreplacing}

% references
\usepackage[natbibapa]{apacite}
\bibliographystyle{apacite}
\renewcommand{\bibsection}{}

% use ampersands instead of "and" for text citations
\AtBeginDocument{\renewcommand{\BBAB}{\&}}

% possessive cites
\makeatletter
\patchcmd{\NAT@test}{\else \NAT@nm}{\else \NAT@nmfmt{\NAT@nm}}{}{}
\DeclareRobustCommand\citepos
  {\begingroup
   \let\NAT@nmfmt\NAT@posfmt
   \NAT@swafalse\let\NAT@ctype\z@\NAT@partrue
   \@ifstar{\NAT@fulltrue\NAT@citetp}{\NAT@fullfalse\NAT@citetp}}
\let\NAT@orig@nmfmt\NAT@nmfmt
\def\NAT@posfmt#1{\NAT@orig@nmfmt{#1's}}
\makeatother

% spaced-out lists
\newenvironment{wideitemize}{\itemize\addtolength{\itemsep}{10pt}}{\enditemize}
\newenvironment{wideenumerate}{\enumerate\addtolength{\itemsep}{10pt}}{\endenumerate}

% replace footnotes with buttons
\usepackage[absolute,overlay]{textpos}
\newcounter{beamerpausessave}
\newcommand{\always}[1]{
    \setcounter{beamerpausessave}{\value{beamerpauses}}
    \setcounter{beamerpauses}{0}
    \pause
    #1 
    \setcounter{beamerpauses}{\value{beamerpausessave}}
    \addtocounter{beamerpauses}{-1}
    \pause
}
\newcommand{\buttons}[1]{\always{
    \begin{textblock*}{\paperwidth}(0.015\textwidth, 1.022\textheight)
        \scriptsize
        #1
    \end{textblock*}
}}
\newcommand{\appendixbuttons}[1]{\always{
    \begin{textblock*}{\paperwidth}(0.015\textwidth, 1.043\textheight)
        \scriptsize
        #1
    \end{textblock*}
}}
\newcommand{\goto}[2]{\hyperlink{#1}{{\color{dark red}$\smalltriangleright$} #2}\hspace{0.5em}}
\newcommand{\goback}[2]{\hyperlink{#1}{{\color{dark red}$\smalltriangleleft$} #2}\hspace{0.5em}}

% custom appendix
\renewcommand{\appendixname}{\texorpdfstring{\translate{Appendix}}{Appendix}}

% change color of cites and URLs
\let\oldcite\cite
\let\oldcitet\citet
\let\oldcitep\citep
\let\oldcitepos\citepos
\let\oldcitetalias\citetalias
\let\oldcitepalias\citepalias
\let\oldurl\url
\def\cite#1#{\citeaux{#1}}
\def\citet#1#{\citetaux{#1}}
\def\citep#1#{\citepaux{#1}}
\def\citepos#1#{\citeposaux{#1}}
\def\citetalias#1#{\citetaliasaux{#1}}
\def\citepalias#1#{\citepaliasaux{#1}}
\def\url#1#{\urlaux{#1}}
\newcommand*\citeaux[2]{{\color{light gray}\oldcite#1{#2}}}
\newcommand*\citetaux[2]{{\color{light gray}\oldcitet#1{#2}}}
\newcommand*\citepaux[2]{{\color{light gray}\oldcitep#1{#2}}}
\newcommand*\urlaux[2]{{\color{light gray}\oldurl#1{#2}}}
\newcommand*\citeposaux[2]{{\color{light gray}\oldcitepos#1{#2}}}
\newcommand*\citetaliasaux[2]{{\color{light gray}\oldcitetalias#1{#2}}}
\newcommand*\citepaliasaux[2]{{\color{light gray}\oldcitepalias#1{#2}}}

% custom math commands
\DeclareMathOperator*{\argmax}{argmax}
\DeclareMathOperator*{\argmin}{argmin}
\renewcommand{\Pr}{\mathbb{P}}
\newcommand{\E}{\mathbb{E}}
\newcommand{\Var}{\mathbb{V}}
\newcommand{\Cov}{\mathbb{C}}
\newcommand{\overbar}[1]{\mkern 1.5mu\overline{\mkern-1.5mu#1\mkern-1.5mu}\mkern 1.5mu}
\newcommand{\abs}[1]{\lvert#1\rvert}
\newcommand{\norm}[1]{\lVert#1\rVert}

% tables
\usepackage{booktabs}
\usepackage{colortbl}
\usepackage{multirow}
\usepackage{makecell}
\arrayrulecolor{dark red}

% custom date
\usepackage{datetime}
\newdateformat{monthyeardate}{\monthname[\THEMONTH] \THEYEAR}

% fix pauses with graphics
\usepackage{../resources/fixpauseincludegraphics}

\usetheme{metropolis}

\usepackage{fancybox}
\usepackage{booktabs} 
\usepackage{dsfont}
\usepackage{stackengine}
\usepackage{tabularx}
\usepackage{dcolumn}
\usepackage{soul}
\usepackage{dsfont}
\usepackage{ulem}


\usepackage{chronosys}
\usepackage{verbatim}
\pagenumbering{arabic}
\usepackage{tabularx}
\usepackage{ragged2e}
\usepackage{mathtools}


\usepackage{multirow,multicol,dcolumn}
\usepackage{amsfonts,amsthm, amsmath, graphics, graphicx, mathtools}



\newenvironment{reference}[2]{% 
  \begin{textblock*}{\textwidth}(#1,#2) 
      \footnotesize\it\bgroup\color{red!50!black}}{\egroup\end{textblock*}} 

\DeclareMathSizes{10}{10}{6}{6} 
\AtBeginSection[]{
  \begin{frame}
  \vfill
  \centering
  \begin{beamercolorbox}[sep=8pt,center,shadow=true,rounded=true]{title}
    \usebeamerfont{title}\insertsectionhead\par%
  \end{beamercolorbox}
  \vfill
  \end{frame}
}

\begin{document}
\title{Conduct}
\author{Chris Conlon}
\institute{Grad IO}
\date{\today}

\frame{\titlepage}


\begin{frame}[plain]{Conduct Testing in Industrial Organization}
Foundational Empirical IO Question: How do we observe data on price and quantity and infer which model of firm behavior generated those outcomes?
\pause
\begin{itemize}
\item Early work: Porter (1983), Bresnahan (1982,1987)
\item Subsequent work defined the ``menu" approach: Nevo (1998, 2001), Villas-Boas (2007) 
\item Recent revival of ``internalization'' parameters: Miller and Weinberg (2017), Crawford, Lee, Whinston, and Yurukoglu (2017), Pakes (2017)
\item Parallel work by: Duarte, Magnolfi, Sølvsten, Sullivan (2022) which test is best (RV). Magnolfi, Quint, Sullivan, Waldfogel (2022) Should we test or estimate?
\item Applications of our test: Starc and Wollman (2022), Scuderi (2022), others?
\end{itemize}
Is conduct testable? Berry and Haile (2014): yes.
\end{frame}


\begin{frame}{Conduct Testing in Industrial Organization}
\begin{itemize}
\item Absent additional restrictions, we cannot generally look at data on $(P,Q)$ and decide whether or not collusion is taking place.
\begin{itemize}
\item You say we started colluding at date $t$, I say we received a correlated shock to $mc$.
\end{itemize}
\item We can make progress in two ways: (1) parametric restrictions on marginal costs; (2) exclusion restrictions on supply.
\begin{itemize}
\item Most of the literature focuses on (1) by assuming something like: $\ln mc_{jt} = x_{jt} \gamma_1 + w_{jt} \gamma_2 + \omega_{jt}$.
\item In principle (2) is possible if we have instruments that shift demand for products but not supply. (These are much easier to come up with than ``supply shifters'').
\end{itemize}
\end{itemize}
\end{frame}



\begin{frame}{A famous plot (Bresnahan 87)}
\includegraphics[width = 6.6cm]{./resources/bres_plot1.png}
\includegraphics[width = 6.9cm]{./resources/bres_plot2.png}\\
Bresnahan (1980/1982) recognized this problem: we need ``rotations of demand''.
\end{frame}


\begin{frame}[plain]{Conduct Testing in Pictures (Berry Haile 2014)}
\begin{center}
\includegraphics[width = 6.5cm]{resources/berryhaile1.png}
\includegraphics[width = 6.5cm]{resources/berryhaile2.png}
\end{center}
\small{Figure 2(ab) from Berry and Haile (2014), Example 1.}
\end{frame}




\begin{frame}{Setup: Notation and Utility}
We begin with a relatively standard BLP-style differentiated products setup.\\

\begin{itemize}
    \item Markets $t$
    \item Products $j$
    \item Data $\chi_t = \{(\textrm{x}_{jt},\textrm{v}_{jt},\textrm{w}_{jt})$ for all $j \in \calJ_t\}$.
    \item Market Shares $\calS_t = [s_{1t}, \ldots, s_{Jt}, s_{0t}]$.
    \item Prices $\symbf{p}_t = [p_{1t}, \ldots, p_{Jt}]$.
    \item Consumers $i$ with demographics $y_{it}$ (income, presence of kids)
\end{itemize}
\end{frame}

\begin{frame}
\frametitle{Testing Conduct: Setup}
\small
We generalize the $\mathcal{H}(\kappa)$ and derive multi-product Bertrand FOCs:
\begin{align*}
\arg \max_{p \in \calJ_f} \pi_f (\symbf{p}) &= \sum_{j \in \calJ_f} (p_j - c_j) \cdot q_j(\symbf{p}) +  \alert{\kappa_{fg} \sum_{j \in \calJ_g} (p_j - c_j) \cdot q_j(\symbf{p})} \\
\rightarrow 0&= q_j(\symbf{p}) + \sum_{k \in (\calJ_f,\calJ_g)} \alert{\kappa_{fg}}\cdot (p_k - c_k) \frac{\partial q_{k}}{\partial p_j}(\symbf{p}) 
\end{align*}
\begin{itemize}
\item Instead of $0$'s and $1$'s we now have $\kappa_{fg} \in [0,1]$ representing how much firm $f$ cares about the profits of $g$.
\begin{itemize}
\item If $f$ and $g$ merge (or fully coordinated) then $\kappa_{fg} =1$
\item Often in the real world firms cannot reach fully collusive profits and $\kappa_{fg} \in (0,1)$.
\item Evidence that $\kappa_{fg} > 0$ is not necessarily evidence of malfeasance, just a deviation from \alert{static Bertrand pricing}
\end{itemize}
\end{itemize}
\end{frame}


\begin{frame}{Testing Conduct: Setup}
\begin{itemize}
\item Recall the $\Delta$ matrix which we can write as $\Delta=\tilde{\Delta}\, \odot \mathcal{H}(\kappa)$, where $\odot$ is the element-wise or Hadamard product of two matrices. 
\begin{itemize}
\item $\tilde{\Delta}$ is the matrix of demand derivatives with $\Delta{(j,k)} = \frac{\partial q_j}{\partial p_k}$ for all elements.
\item $\mathcal{H}(\kappa)=\kappa_{fg}$ for products owned by $(f,g)$ where $\kappa_{ff}=1$ always.
\end{itemize}
\item Mergers are about changing $0$'s to $1$'s in the $\mathcal{H}(\kappa)$ matrix.
\item Matrix form of FOC: $q(\symbf{p}) = \Delta(\symbf{p},\kappa)\cdot(\symbf{p}-\symbf{mc})$
\item $\symbf{mc} =  \symbf{p} - \underbrace{\Delta(\symbf{p}, \theta_2, \kappa)^{-1} s(\symbf{p})}_{\eta(\symbf{p},\symbf{s},\theta_2,\kappa)}$
where $\eta_{jt}$ is the markup.
\end{itemize}
\end{frame}

\begin{frame}
\frametitle{Reasons for Deviations from Static Bertrand}
\small
\begin{description}
\item[Biased estimates of own and cross price derivatives:] For anything to work, you have correct estimates of $\tilde{\Delta}$. My prior is most papers \alert{underestimate} diversion ratios for close substitutes.
\item[Vertical Relationships:] Who sets supermarket prices? Just the retailer? Just the manufacturer? Some combination of both? Retailers tend to \alert{soften} downstream price competition.
\item[Faulty Timing Assumptions:] Bertrand is a simultaneous move pricing game. Lots of alternatives (Stackelberg leader-follower, Edgeworth cycles, etc.).
\item[Dynamics and Dynamic Pricing:] Forward looking firms or consumers might not set static Nash prices. [e.g. Temporary Sales, Switching Costs, Network Effects, etc.]
\item[Unmodeled Supergame:] Maybe firms are legally tacitly colluding, higher prices might be about what firms believe will happen in a price war.
\end{description}
\end{frame}


\begin{frame}{Simultaneous Problem}
Assume additivity, and write in terms of structural errors:
\begin{align*}
\delta_{jt}(\calS_t,\widetilde{\theta}_2) + \alpha p_{jt} &= h_d(\textrm{x}_{jt}, \, \alert{\textrm{v}_{jt}},\theta_1)  + \xi_{jt} \\
 f \left( p_{jt} - \eta_{jt}(\symbf{p},\symbf{s},\theta_2,\kappa) \right) &= h_s(\textrm{x}_{jt}, \alert{\textrm{w}_{jt}},\theta_3) + \omega_{jt}
\end{align*}
\vspace{-.4cm}
\begin{itemize}
 \item To simplify slides we let $f(x)=x$ (often $f(x) =\log(x))$ but we can put that in $h_s(\cdot)$.
\item $h(\cdot)$ are often just linear relationships like: $\theta_1 [\textrm{x}_{jt}, \textrm{v}_{jt}]$.
\item Endogeneity Problem: $p_{jt}$ and $\eta_{jt}$ are functions of $(\symbf{\xi},\symbf{\omega})$.
\item $(\theta_2, \kappa)$ parameters that determine markups
\end{itemize}
\end{frame}

\begin{frame}
\frametitle{Approach \#1: Demand Side}
1. Estimate $\theta_2$ from demand alone.
\begin{align*}
\delta_{jt}(\calS_t,\widetilde{\theta}_2) + \alpha p_{jt} &= h_d(\textrm{x}_{jt}, \, \alert{\textrm{v}_{jt}},\theta_1)  + \xi_{jt} \\
E[\xi_{jt} | \textrm{x}_{t}, \textrm{v}_{t}, \textrm{w}_{t}] &=0
\end{align*}
2. Recover marginal costs $\widehat{\symbf{mc}} = \symbf{p} + \symbf{\eta}$
\begin{align*}
\symbf{\eta}(\symbf{p},\symbf{s},\theta_2,\kappa) \equiv \left(\mathcal{H}(\kappa).*\tilde{\Delta}(\symbf{p},\theta_2) \right)^{-1} q(\symbf{p})
\end{align*}
\vspace{-0.2cm}
Challenges:
\begin{itemize}
\item Given $[\symbf{q},\symbf{p},\tilde{\Delta},\mathcal{H}(\kappa)]$ I can always produce a vector of marginal costs $\symbf{mc}$ that rationalizes what we observe. [ie: $J$ equations $J$ unknowns].
\item Nonparametrically we cannot identify $\kappa$ without more restrictions (!).
\end{itemize}
\end{frame}



\begin{frame}\frametitle{What do people do?}
Maybe some vectors of $\symbf{mc}$ look less ``reasonable'' than others.
\begin{itemize}
\item Marginal costs $\leq 0$ seem problematic.
[Might just be that your estimates for demand are too inelastic...]
\item or I have a parametric model of MC in mind. 
\begin{align*}
 f \left( p_{jt} - \eta_{jt}(\symbf{p},\symbf{s},\theta_2,\kappa) \right) &= h_s(\textrm{x}_{jt}, \textrm{w}_{jt},\theta_3) + \omega_{jt}\\
 E[\omega_{jt} | \textrm{x}_{t}, \textrm{w}_{t}, \alert{\textrm{v}_{t}}]&=0
\end{align*}
\item Can test that model with GMM objective of $mc_{jt}$ on regressors.
\item Maybe marginal costs cannot deviate too much within product from period to period. (We can write these as moment restrictions too).
\end{itemize}
\end{frame}

\begin{frame}
\frametitle{Approach \#2: Simultaneous Supply and Demand}
Estimate $\theta_2$ using both supply and demand. The fit of my supply side will also inform my demand parameters, particularly $\alpha$ the price coefficient. [BLP 95 used this for additional power with lots of random coefficients and potentially weak instruments].
\begin{align*}
\delta_{jt}(\calS_t,\widetilde{\theta}_2) + \alpha p_{jt} &= h_d(\textrm{x}_{jt}, \, \alert{\textrm{v}_{jt}},\theta_1)  + \xi_{jt} \\
 f \left( p_{jt} - \eta_{jt}(\symbf{p},\symbf{s},\theta_2,\kappa) \right) &= h_s(\textrm{x}_{jt}, \alert{\textrm{w}_{jt}},\theta_3) + \omega_{jt}
\end{align*}
Challenges:
\begin{itemize}
\item Should I try to estimate $\kappa$? or just compare objective values at $\kappa_{fg}\in\{0,1\}$?
\item Am I testing conduct? Or am I testing the functional form for my supply model?
\item Will a missing IV/restriction change whether or not I believe firms are colluding?
\end{itemize}
\end{frame}

% \begin{frame}
% \frametitle{What is Excluded?}
% Berry and Haile (2014) discuss \alert{non-parametric} identification of conduct via exclusion restrictions:
% \begin{itemize}
% \item We used exlcuded cost shifters $\alert{w_{jt}}$ as IV for demand. We can use excluded demand shifters $\alert{v_{jt}}$ as IV for supply.
% \begin{itemize}
% \item Probably easier to find these. Rich people are less price sensitive but not more costly to sell to (demographics, seasonality, etc.).
% \item Well-documented geographic persistence in preferences unrelated to costs.
% \end{itemize}
% \end{itemize}
%  If we take the structural interpretation seriously any $\alert{v_{jt}}$ should show up in the utility equation to be \alert{relevant} (!).
% \end{frame}

% \begin{frame}
% \frametitle{What else is Excluded?}
% BLP style instruments (characteristics of other goods)
% \begin{itemize}
% \item $f(x_{-j})$: BLP or GH style instruments (how many similar cars to me?).
% \item $w_{-j}$ Cost shifters for other products (Price of Rice for Corn Flakes, Price of Corn for Rice Krispies).
% \item $v_{-j}$ Demand shocks for similar products (Advertising? Product Recalls?)
% \item $\kappa$ parameters or $\kappa$ weighted diversion ?
% \end{itemize}
% An ideal restriction should \alert{not} shift marginal costs under the true model of conduct $\kappa$ but could potentially shift marginal costs under the alternative $\kappa$ (this is relevance).
% \end{frame}


% \begin{frame}{Things that don't work}
% \begin{itemize}
% \item $\xi_{jt}$ only makes sense if you believe $Cov(\xi_{jt},\omega_{jt})=0$.
% \begin{itemize}
%   \item MacKay Miller exploit this to estimate demand without IV? Is this a good idea(?)
% \end{itemize}
% \item $p_{j,t,-s}$ (Hausman instruments) same good in other markets: pick up cost shocks (but could pick up changes in conduct!). 
% \item If it isn't in one of our equations: does it have anything to do with demand or supply?
% \end{itemize}
% \end{frame}


\begin{frame}{Estimation vs. Menus}
There are two ways to think about conduct:
\begin{enumerate}
\item Using moment conditions to estimate $\widehat{\kappa}$ or $\mathcal{H}(\kappa)$ directly.
\begin{itemize}
\item Often with a small number of parameters (ie: $\kappa_{fg}=0$ except for firms I know are in a cartel).
\item Can be challenging to tell similar values of $\kappa_{fg}$ apart (under-powered).
\end{itemize}
\item ``Menu Approach''
\begin{itemize}
\item  Nevo (Economics Letters 1998)
\item Bresnahan (1987)
\item Compare some goodness-of-fit critieria across assumed values of $\kappa$ (Bertrand vs. Collusion)
  \end{itemize}
\item Rejecting (or failing to reject) a single model.
  \end{enumerate}
\end{frame}


\begin{frame}{Testing a single model of $\kappa$}
Put the $\eta_{jt}$ on the RHS and test whether $\lambda=1$:
\begin{align*}
 p_{jt} = h_s(\textrm{x}_{jt}, \alert{\textrm{w}_{jt}},\theta_3) + \lambda \cdot \eta_{jt}\left(\symbf{p},\symbf{s},\theta_2,\kappa \right)+  \omega_{jt} \text{ with }
 E[\omega_{jt} | \textrm{x}_{t},\textrm{w}_{t},\textrm{z}_{t}]=0s
\end{align*}
\vspace{-0.25cm}
\begin{itemize}
\item We are basically running 2SLS with IV for the endogenous $\eta_{jt}$
\item ``Informal'' test of Villas Boas (2007): $\mathbb{E}[\omega_{jt} | \textrm{x}_{jt},\alert{\textrm{w}_{jt}},\alert{\eta_{jt}}]=0$.
\begin{itemize}
\item Considers different forms of $f(\cdot)$: linear, exponential, logarithmic.
\item Not sure the published paper includes these results (?) WP does?
  \end{itemize}
\item Pakes (2017) uses Wollman (2018) data and BLP IV $\mathbb{E}[\omega_{jt} | x_{jt},w_{jt},f(x_{-j})]=0$.
\item $\lambda \neq 0$ is hard to interpret.
\end{itemize}
\end{frame}


\begin{frame}{Pakes (2018)/Wollman (2017) Regressions: Heavy Trucks}
\begin{center}
\includegraphics[height=0.9\textheight]{./resources/wollman_regression.png}
\end{center}
\end{frame}

\begin{frame}{Single Model Regressions}
These are somewhat reassuring:
\begin{itemize}
\item $\lambda\approx 1$ for multiproduct-oligopoly
\item Fit is pretty good $R^2 > 0.8$ and $R^2 > 0.5$ for within vehicle regressions (not shown).
\item As a behavioral model, multiproduct demand estimation seems successful.
\item But, do we know that an alternative $\mathcal{H}(\kappa)$ would have a $\lambda \neq 1$ or a lower $R^2$, and if so how low before we can ``reject'' the model?
\end{itemize}
\end{frame}

\begin{frame}{Goodness of Fit Tests}
Another idea (Bonnet and Dubois, Rand 2010) runs the following regression:
\begin{align*}
\log \left( p_{jt} - \eta_{jt}(\symbf{p},\symbf{s},\widehat{\theta}_2,\kappa) \right) &= h_s(x_{jt}, \alert{w_{jt}},\theta_3) + \omega_{jt}
\end{align*}
\begin{itemize}
\item Run a regression for each $\kappa$ and obtain $Q(\kappa)=\sum_{jt} \widehat{\omega}_{jt}^2$
\item Employ the \alert{non nested test} of Rivers and Vuong (2002). Why?
\item Working out the distribution of $Q(\kappa_1) - Q(\kappa_2)=T(\kappa_1,\kappa_2)$ is the hard part.
\item Also this is OLS (or NLLS) and there are no instruments or \alert{exclusion restrictions} for the supply side. Presumably we could add some and do GMM? (I think this is the ``formal'' test of Villas Boas (ReStud 2007)).
\end{itemize}
\end{frame}

\begin{frame}{Recap}
\small
So far three approaches to exploit $ E[\omega_{jt} | x_{t},w_{t},z_{t}]=0$
\begin{enumerate}
\item Put the markup on RHS and instrument for it to test $\lambda=1$ (Wald)
\begin{align*}
 p_{jt} &= h_s(x_{jt}, \alert{w_{jt}},\theta_3) + \lambda \cdot \eta_{jt}\left(\symbf{p},\symbf{s},\widehat{\theta_2},\kappa \right)+  \omega_{jt}
\end{align*}
\item Put the markup on LHS assuming $\lambda=1$ and test goodness of fit of supply equation (Anderson Rubin)
\begin{align*}
 p_{jt} -\eta_{jt}\left(\symbf{p},\symbf{s},\widehat{\theta_2},\kappa \right)&= h_s(x_{jt}, \alert{w_{jt}},\theta_3) +  \omega_{jt}
\end{align*}
\item Estimate supply and demand simultaneously $[\theta_1,\theta_2,\theta_3]$ and compare goodness of fit for different $\kappa$. (Likelihood Ratio)
\end{enumerate}

\end{frame}

% \begin{frame}{Simultaneous Problem: Menu Approach}
% Assume two models of conduct (correct: $\kappa_0$) (incorrect: $\kappa_1$)
% \begin{align*}
% %\label{eq:both_mc}
% f(p_{jt} -\eta_{jt}(\kappa_0))= h(\textrm{x}_{jt},\textrm{w}_{jt};\theta_3^0)+\omega_{jt}^{0},\\
% f(p_{jt} -\eta_{jt}(\kappa_1))= h(\textrm{x}_{jt},\textrm{w}_{jt};\theta_3^1)+\omega_{jt}^{1}.
% \end{align*}
% Write things in terms of the markup difference:
% \begin{align*}
% p_{jt} -\eta_{jt}(\kappa_1)= h(x_{jt},w_{jt};\theta_3)+ \overbrace{\lambda \cdot  \Delta \eta_{jt}(\symbf{p},\symbf{s},\theta,\kappa_0,\kappa_1) +   \omega_{jt}}^{\widetilde{\omega_{jt}}}
% \end{align*}
% Tempting idea: run the above regression and test if $\lambda=0$.
% \begin{itemize}
% \item True model $\lambda=0$, alternate model $\lambda \neq 0$.
% \item True model will satisfy $E[\widetilde{\omega}_{jt} | x_{t}, \alert{w_{t}}, \alert{v_{t}}]=0$
% \item $\eta_{jt}$ is \alert{endogenous}: it depends on everything including $(\xi,\omega)$.
% \end{itemize}
% \end{frame}


% \begin{frame}{A subtle solution}
% \begin{itemize}
% \item Berry Haile 2014 tell us we need \alert{marginal revenue shifters} to act as \alert{exclusion restrictions}.
% \item Needs to be uncorrelated with $p_{jt}-\eta_{jt}(\kappa_0)$ but correlated with $p_{jt}-\eta_{jt}(\kappa_1)$
% \begin{itemize}
% \item If my marginal cost is correlated with marginal costs of other products or ``closeness of competitors'', I've got the wrong conduct assumption!
% \end{itemize}
% \item We need an instrument for $\Delta \eta_{jt}(\symbf{p},\symbf{s},\theta,\kappa_0,\kappa_1)$
% \begin{itemize}
% \item Maybe not so hard since it is basically a function of everything.
% \item Cannot have a direct effect on $mc_{jt}$ (exclusion restriction).
% \end{itemize}
% \end{itemize}
% \end{frame}



% \begin{frame}{Backus, Conlon, Sinkinson (2020)}
% What would a really good instrument look like?
% \begin{itemize}
% \item Chamberlain (1987) style optimal IV for $\kappa_{fg}$ would be $E\left[\frac{\partial \eta_{jt}(\theta_2,\symbf{s},\symbf{p},\kappa)}{\partial \kappa_{fg}} | x_{t}, w_{t}, v_{t}\right]$
% \begin{itemize}
% \item But infeasible without knowledge of $(\kappa,\xi,\omega)$!
% \item We could try to recover the infeasible estimate and project it onto $(x_t,w_t,v_t)$ (note: lack of $j$ subscripts!)
% \end{itemize}
% \item Menu approach: could look at discrete analogue: $E\left[\Delta \eta_{jt}(\kappa_1,\kappa_0,\theta_2,\symbf{s},\symbf{p}) | x_{t}, w_{t}, v_{t}\right]$
% \begin{itemize}
% \item I would need to know $\kappa_1,\kappa_0$.
% \item Still infeasible but could run a first-stage regression
% \end{itemize}
% \end{itemize}
% \end{frame}



% \begin{frame}{Backus, Conlon, Sinkinson (2020)}
% What would a really good instrument look like?
% \begin{itemize}
% \item Chamberlain (1987) style optimal IV for $\kappa_{fg}$ would be $E\left[\frac{\partial \eta_{jt}(\theta_2,\symbf{s},\symbf{p},\kappa)}{\partial \kappa_{fg}} | x_{t}, w_{t}, v_{t}\right]$
% \begin{itemize}
% \item But infeasible without knowledge of $(\kappa,\xi,\omega)$ so we take expectation over exogenous variables.
% \item We could try to recover the infeasible estimate and project it onto $(x_t,w_t,v_t)$ (note: lack of $j$ subscripts!)
% \end{itemize}
% \item Menu approach: could look at discrete analogue: $E\left[\Delta \eta_{jt}(\kappa_1,\kappa_0,\theta_2,\symbf{s},\symbf{p}) | x_{t}, w_{t}, v_{t}\right]$
% \begin{itemize}
% \item I would need to know $\kappa_1,\kappa_0$.
% \item Still infeasible but could run a first-stage regression
% \end{itemize}
% \end{itemize}
% \end{frame}


% \begin{frame}{Backus, Conlon, Sinkinson (2020)}
% \footnotesize
% Our procedure ((1)+(2) can be done separately)
% \begin{enumerate}
% \item Run OLS to obtain $\widehat{\omega}_1,\widehat{\omega}_2$ for $(\kappa_1,\kappa_2)$
% \begin{align*}
% \log \left( p_{jt} - \eta_{jt}(\symbf{p},\symbf{s},\widehat{\theta}_2,\kappa) \right) = h_s(x_{jt}, w_{jt},\theta_3) + \omega_{jt}
% \end{align*}
% \item Recover $\Delta \widehat{\eta}_{jt}(\kappa_1,\kappa_2)$ via nonparametric regression/machine-learning
% \begin{align*}
% \Delta \widehat{\eta_{jt}}(\kappa_1,\kappa_2) = E \left[\Delta \eta_{jt}(\kappa_1,\kappa_2) | z_{t},w_{t},x_{t} \right]
% \end{align*}
% \item Compute the violations of the moment condition
% $Q\left(\kappa^{m}\right)=\left(n^{-1} \sum_{j, t} \hat{\omega}_{j t}^{m} \cdot \widehat{\Delta \eta}_{j t}\right)^{2}$
% \item Compute the test statistic: $T=\frac{\sqrt{n}\left(Q\left(\kappa^{1}\right)-Q\left(\kappa^{2}\right)\right)}{\hat{\sigma}}$ and bootstrap the standard error.
% \end{enumerate}
% Techincally we should \alert{sample split} and estimate the the regressions on \alert{independent} samples.
% \end{frame}


% \begin{frame}{Backus, Conlon, Sinkinson (2020) [Alternative]}
% \begin{align*}
%  p_{jt} -\eta_{jt}\left(\symbf{p},\symbf{s},\widehat{\theta_2},\kappa \right)&= h_s(x_{jt}, \alert{w_{jt}},\theta_3) +  \omega_{jt}
% \end{align*}
% \begin{itemize}
% \item We can also directly test violations of $E[\omega_{jt} | x_t, v_t, w_t]$ by comparing resulting CUE or GEL (or GMM) objective values.
% \item Probably want to include approximate optimal IV $E\left[\frac{\partial \eta_{jt}(\theta_2,\symbf{s},\symbf{p},\kappa)}{\partial \kappa_{fg}} | x_{t}, w_{t}, v_{t}\right]$ in instrument set.
% \end{itemize}
% \end{frame}









\section{Backus Conlon Sinkinson (2022)}




\begin{frame}{Basic Setup}
We start with marginal revenue and marginal cost (unobserved $\omega$, observed $h(\cdot)$)
\begin{align*}
\psi_{jt}^{m} &= mc_{jt} \\
p_{jt} -\eta_{jt}^m &= h_s(\textrm{x}_{jt},\textrm{w}_{jt})  + \omega_{jt}^m
\end{align*}
\begin{itemize}
\item Let's be vague/flexible with $h_s(\cdot)$ for now, but I don't know the production function.
\item Assume: Demand and hence $\eta_{jt}^{m}$ are \alert{known (given conduct)}.\\
\item Idea $(\eta^{A},\eta^{B})$ are monopoly/perfect competition or Cournot/Bertrand.
\end{itemize}
\end{frame}



\begin{frame}{The Question}
Two competing markups $(\eta_{jt}^A, \eta_{jt}^B)$: which fits the data better?\\
(both may be misspecified)
\begin{align*}
p_{jt} = h_s(\textrm{x}_{jt},\textrm{w}_{jt}) + \tau\, \eta_{jt}^A + (1-\tau)\,\eta_{jt}^B + \omega_{jt}
\end{align*}
Model is defined by a conditional moment restriction $\mathbb{E}[\omega_{jt}  | z_{jt}^s]=0$
\begin{itemize}
\item $H_0: \tau =1 \text{ vs } H_a: \tau = 0$
\item This is a \alert{model selection} problem or a \alert{non nested testing} problem.
\begin{itemize}
\item We might want to compare more than two alternatives (too bad).
\end{itemize}
\item Obvious endogeneity problem with $\eta_{jt}$!
\end{itemize}
\end{frame}


\begin{frame}{Testing Environment}
Compare violations of unconditional moments under $(\eta_{jt}^A, \eta_{jt}^B)$ and $A(z_{jt}^s)$:
\begin{align*}
p_{jt} -  \eta_{jt}^A = h_s(\textrm{x}_{jt},\textrm{w}_{jt}) + \omega_{jt}^{A}\\
p_{jt} -  \eta_{jt}^B = h_s(\textrm{x}_{jt},\textrm{w}_{jt}) + \omega_{jt}^{B}
\end{align*}
\pause
Which gives us
\begin{align*}
g_A = \frac{1}{N} \sum_{jt} \omega_{jt}^{A}\, A(z_{jt}^s), &\quad
g_B =\frac{1}{N} \sum_{jt}  \omega_{jt}^{B}\, A(z_{jt}^s)\\
Q_m &= g_m'\, W_m\, g_m
\end{align*}
Now consider a \alert{Rivers Vuong (2002)} type test $T_{RV} = \sqrt{n} \left(\frac{Q_A - Q_B}{\sigma_{Q_A - Q_B}}\right) \sim N(0,1)$
$H_0: Q_A - Q_B=0$ vs. $H_A: Q_A > Q_B$ or $Q_A < Q_B$.\\
Getting the SD of the difference is hard $\rightarrow$ bootstrap 
\end{frame}



\begin{frame}{Comparison to Literature}
\begin{itemize}
\item Bresnahan (1987): Did LR test to determine collusion vs. competition in 1955 automobile price war
\begin{itemize}
\item No IV, errors were measurement in $P,Q$.
\end{itemize}
\item Bonnet and Dubois (2010): RV test
\begin{itemize}
\item But no IV -- maximum likelihood with normally distirbuted $\omega_{jt}$'s.
\end{itemize}
\item Villas Boas (2007): Cox test to determine double marginalization or not in yogurt
\begin{itemize}
\item GMM objective, unclear what if any IV are used.
\item Need to ``know'' the true model.
\end{itemize}
\item Duarte, Magnolfi, Solvsten, Sullivan (2022): RV beats Cox pretty badly in Monte Carlo.
\end{itemize}
\end{frame}




\begin{frame}[plain,label=mainresults]{Main Results: These are $N(0,1)$}
\begin{center}
\scalebox{0.55}{\begin{tabularx}{500pt}{l*4             {>{\Centering}X}}\toprule
{} &  Others' Cost &  Demographics &  BLP Inst. &  Dmd. Opt. Inst. \\
\midrule \multicolumn{1}{c}{Own Profit Max vs.}&             \multicolumn{4}{c}{Panel 1: $A(\symbf{z}_t)=\symbf{z}_t,$ linear $h_s(\cdot)$ }\\                 \cmidrule(lr){1-1} \cmidrule(lr){2-5}
%Single Product                            &        3.0840 &        1.1230 &     0.9976 &           0.6859 \\
Common Ownership                          &       -4.3410 &       -1.1966 &     0.5047 &          -1.2552 \\
Double Marginalization                    &        2.1922 &        1.0055 &    -0.0412 &           7.0897 \\
Double Marginalization + CO &       -0.8262 &        0.6892 &     0.1428 &           6.9320 \\
Perfect Competition                       &        3.2995 &        0.5194 &     0.7355 &           3.7223 \\
Monopolist                                &       -2.2264 &       -1.0528 &    -0.4525 &          -0.9202 \\

 \midrule 

\multicolumn{1}{c}{Own Profit Max vs.}& \multicolumn{4}{c}{Panel 2:             $A(\symbf{z}_t)=\mathbb{E}[\Delta \eta^{12}|\symbf{z_t}]$, linear $h_s(\cdot)$ and $g(\cdot)$}\\                            \cmidrule(lr){1-1} \cmidrule(lr){2-5}
%Single Product                            &        1.4264 &        0.5795 &     0.6662 &           1.2368 \\
Common Ownership                          &       -2.3044 &       -0.5105 &    -0.0384 &          -1.6133 \\
Double Marginalization                    &        0.8644 &        0.4421 &    -0.5311 &           3.3367 \\
Double Marginalization + CO &       -0.9382 &       -0.2389 &    -0.3684 &          -0.0045 \\
Perfect Competition                       &        0.7164 &        0.6135 &    -0.1080 &          -0.3151 \\
Monopolist                                &       -0.8577 &       -0.4002 &    -0.3868 &          -1.2339 \\

 \midrule 

\multicolumn{1}{c}{Own Profit Max vs.}& \multicolumn{4}{c}{Panel 3:             $A(\symbf{z}_t)=\mathbb{E}[\Delta \eta^{12}|\symbf{z_t}]$, random forest $h_s(\cdot)$ and $g(\cdot)$}\\                    \cmidrule(lr){1-1} \cmidrule(lr){2-5}
%Single Product                            &        4.8400 &        5.0700 &     5.1738 &           5.4990 \\
Common Ownership                          &       -3.3777 &       -3.2509 &    -3.7130 &          -4.0256 \\
Double Marginalization                    &       -5.9699 &       -9.9547 &    -6.5789 &          -7.8269 \\
Double Marginalization + CO &       -5.9264 &       -6.1550 &    -6.5231 &          -7.4760 \\
Perfect Competition                       &       -4.0468 &       -6.1901 &    -5.1494 &          -6.3484 \\
Monopolist                                &       -3.4972 &       -4.0070 &    -3.4358 &          -3.7495 \\
\bottomrule
\end{tabularx}
}
\end{center}
\end{frame}
% \hyperlink{morepanels}{\beamerskipbutton{additional specifications}}\hyperlink{mcregs}{\beamerskipbutton{marginal cost regressions}}
%\hyperlink{step2regs}{\beamerskipbutton{step 2 regressions}} \hyperlink{pwregs}{\beamerskipbutton{Pakes-Wollmann regressions}}

\begin{frame}[plain]{An Internalization Parameter}
Let $\kappa$ represent the weight a firm places on competitors and $\tau$ the internalization of those weights.
 \begin{equation*}
 arg\max_{p_j \,:\, j \in \calJ_f} \sum_{j \in \calJ_f} (p_j - mc_j) \cdot s_j(\symbf{p})+
 \sum_{g\neq f} \alert{\tau} \kappa_{fg} \sum_{j \in \calJ_g} (p_k - mc_k) \cdot s_k(\symbf{p})
 \end{equation*}
Now, 
\begin{itemize}
\item $\tau = 0$ implies own-profit maximization
\item $\tau = 1$ implies common ownership pricing
\item $\tau$ in between is..? Agency?
\end{itemize}
We test $\tau \in (0.1, \ldots, 0.9)$ against own-profit maximization.
\end{frame}

\begin{frame}[plain]{Internalization Parameter Results}
\begin{center}
\includegraphics[width=10cm]{resources/tau_figure2.pdf}
\end{center}
\end{frame}


% \begin{frame}{Setup: Demand (BLP95, BLP99, Nevo 2000, ...)}
% Consumer $i$ makes discrete choice in market $t$ and receives utility for choice $j$:
% \begin{align*}
% u_{ijt} &=  h_d(\textrm{x}_{jt}, \textrm{v}_{jt}; \theta_1) - \alpha\, p_{jt} + \lambda \, \log(\text{ad}_{jt}) + \xi_{jt}  + \mu_{ijt}(\textrm{x}_{jt}, y_i; \widetilde{\theta}_2) + \varepsilon_{ijt}
% \end{align*}
% \pause
% Following Berry, Levinsohn, Pakes, we fix $\theta_2=[\widetilde \theta_2,\alpha, \lambda]$ and invert system of equations for $\calS_t$ to get mean utilities:
% \begin{align*}
% \sigma_{j}^{-1}(\calS_t, \chi_t, \symbf{p_t},\mathrm{y_t}; \widetilde{\theta}_2) + \alpha\, p_{jt} - \lambda \, \log(\text{ad}_{jt}) &= h_d(\textrm{x}_{jt}, \textrm{v}_{jt}; \theta_1) + \xi_{jt} 
% \end{align*}
% Can estimate $[\widehat{\theta_1}, \widehat{\theta_2}]$ using the CMR $\mathbb{E}[\xi_{jt} | z_{jt}^d]=0$.
% \end{frame}

% \begin{frame}{Setup: Supply and Conduct}
% Assume that we know $\widehat{\theta_2}$ from demand:
% \begin{align*}
% p_{jt} - \eta_{jt}(\calS_t, \chi_t, \symbf{p_t},\mathrm{y_t}; \widehat{\theta}_2) = h_s(\mathrm{x}_{jt}, \mathrm{w}_{jt}, q_{jt}; \theta_3) + \omega_{jt} \quad &\text{ with } \mathbb{E}[\omega_{jt} | z_{jt}^s]=0
% \end{align*}
% \begin{itemize}
% \item The model is defined by \alert{conditional moment restrictions}.
% \item Testing conduct is typically about detecting violations of the CMR for supply.
% \item We follow the \alert{sequential approach} and estimate $[\widehat \theta_1,\widehat \theta_2]$ from demand and then test supply separately.
% \item The $\theta_3$ parameters are \alert{nuisance parameters}, we don't care what they are, we just want to measure violations of moments.
% \end{itemize}
% \end{frame}


\begin{frame}{Setup: Challenges}
The true model for markups (conduct) will satisfy the CMR: $\mathbb{E}[\omega_{jt} | z_{jt}^s]=0$
\begin{align*}p_{jt} - \eta_{jt}^{(m)} &= h_s(\textrm{x}_{jt}, \textrm{w}_{jt}; \theta_3) + \omega_{jt}
\end{align*}
Goal is test two competing markups $\eta_{jt}^{(A)},\eta_{jt}^{(B)}$, but there are challenges:
\pause
\begin{enumerate}
\item Test will depend on how we choose \alert{unconditional moment restrictions} $\mathbb{E}[\omega_{jt} \cdot \, A(z_{jt}^s)]=0$
\pause
\item Test may depend on how we specify $h_s(\cdot)$
\begin{itemize}
\item All tests are basically joint tests of the specification for \alert{observed marginal costs} and the  \alert{exclusion restriction}.
\item Villas Boas (2007) tries log, linear, exponential in $x \beta $
\end{itemize}
\pause
\item Choice of $\eta_{jt}^{(m)}$ will affect our choice of \alert{weighting matrix} and thus the test. (Hall Pelletier (2011))
\end{enumerate}
\end{frame}


\begin{frame}[plain,label=chamberlain]{A Brief Aside: Chamberlain (1987) in a Slide}
What contains as much information as the CMR $\mathbb{E}[\omega|z_{jt}^s]$ and moments of the form $ \mathbb{E}[\omega_{jt}\cdot A(z_{jt}^s)]. $
\begin{itemize}
\item For linear models $A(z_{jt}^s) = z_{jt}^s$ is generally without loss.
\item For nonlinear models, Chamberlain (1987) shows that the efficient estimator uses
$$A(z_{jt}^s) = \mathbb{E}\left[\frac{\partial \omega_{jt}}{\partial \theta}|z_{jt}^s\right]$$
\item That is not too helpful (its a function of the unknown $\theta$).
\item Much of the follow-up work has been about feasible approximations to this ``optimal instrument" (e.g., Newey 1990)
\end{itemize}
For us a similar concern arises, but it is about \alert{power} to distinguish conduct models rather than \alert{efficiency} of estimation.
%\hyperlink{label=cmrbasics}{\beamerskipbutton{back}} 
\end{frame}



\begin{frame}[plain,label=innovation]{Our Idea: Motivation \#1 (Optimal IV)}
The model is given by
\begin{align*}
p_{jt} &= h_s(\textrm{x}_{jt},\textrm{w}_{jt},\theta_3) + \tau \cdot \eta^A_{jt} + (1-\tau) \cdot \eta^B_{jt} + \omega^m_{jt}\\
& \text{  where  }  H_0: \tau=1 \text{ and } H_a: \tau = 0
\end{align*}

\begin{itemize}
\item The optimal IV in the Chamberlain (1987) sense is given by $\mathbb{E}\left[\frac{\partial \omega_{jt}}{\partial \tau}|z_{t} \right]= \mathbb{E}\left[\eta^A_{jt}-\eta_{jt}^B|z_{t}\right]$.
\item In words: The IV need to predict the \alert{difference in markups}\\ (beyond observed $h_s(\textrm{x}_{jt},\textrm{w}_{jt},\theta_3)$).
\end{itemize}
\end{frame}




\begin{frame}[plain,label=misspecification]{Our Idea: Motivation \#2 (Misspecification)}
Index the \alert{true} model by $0$. Then,
$$ p_{jt} -\eta^0_{jt}= h_s(x_{jt},w_{jt}) + \omega^0_{jt}.$$
To motivate a useful test, we ask what happens when we estimate supply with the \alert{wrong} conduct model (1):
\pause
$$p_{jt} -\eta_{jt}^1 = h_s(x_{jt},w_{jt}) + \underbrace{\underbrace{\eta^0_{jt} - \eta^1_{jt}}_{\equiv \Delta \eta_{jt}^{0,1}} +  \omega_{jt}^{0}}_{\omega_{jt}^{1}}.$$
\begin{itemize}
\item Misspecifying conduct introduces an omitted variable: the difference in markups.
\item Our test is premised on detection of this omitted variable.
\end{itemize}
% \hyperlink{advantages}{\beamerskipbutton{back}} 
\end{frame}


\begin{frame}[plain,label=innovation]{Our Innovation: How does this help?}
\begin{small}
The model is given by
$$p_{jt} - \eta^m_{jt} = h_s(\cdot) +  \omega^m_{jt} \text{,   and  } \mathbb{E}[\omega_{jt}^{(m)}\cdot A(z_t)] = 0.$$

We suggest $A(z_t) = \mathbb{E}[\eta^1_{jt}-\eta_{jt}^2|z_{t}]$; several advantages:
\pause
\begin{itemize}
% \item What is the point of instruments for testing? To explain the \alert{difference in markups}
\item Reduces potentially many moments ($\mathbb{E}[\omega_{jt}' z_t]=0$) to a single, scalar moment. No need for a weighting matrix, or associated problems.
\item Testing is reduced to two prediction exercises: $\mathbb{E}[\eta^1_{jt}-\eta_{jt}^2|z_{t}]$ and $\widehat \omega_{jt}^{(m)}$.
\item Show in the paper that this leads to the most powerful test (maximizes distance between two GMM objective functions conditional on weight matrix).
\item Downside: Our choice of instrument is \alert{model specific}! UMP is not going to happen.
\end{itemize}
\end{small}
\end{frame}





\begin{frame}{Possible Exclusion Restrictions}
We are looking for variables which affect \alert{demand but not supply}:
\begin{align*}
\sigma_{j}^{-1}(\calS_t, \symbf{p_t},\alert{\mathrm{y_t}}, \mathrm{x}_t,  \mathrm{v}_t, \widetilde{\theta}_2) 
&= h_d(\mathrm{x}_{jt},\alert{ \mathrm{v}_{jt}}; \theta_1) - \alpha\, p_{jt} + \lambda \, \log(\text{ad}_{jt})+ \xi_{jt} \\
p_{jt} - \eta_{jt}(\calS_t, \symbf{p_t}; \theta_2, \mathcal{H}_t(\kappa))
&= h_s(\textrm{x}_{jt}, \alert{\mathrm{w}_{jt}}; \theta_3) + \omega_{jt} 
\end{align*}
Things we use:
\begin{itemize}
    \item Obvious choice: $\mathrm{v}_{jt}$ (things like product recalls are relatively weak)
    \item Demographics (enter nonlinearly): $\mathrm{y}_t$ (chain-level income works well)
    \item Characteristics of other goods: $f(\mathrm{x}_{-j,t})$ (BLP instruments).
    \item Characteristics of other goods: $\mathrm{w}_{-j,t}$ (commodity price of oats for Rice Krispies)
\end{itemize}
\pause
Things we don't use:
\begin{itemize}
    \item Unobserved demand shocks $\xi_{jt}$ (see MacKay Miller 2020 for $Cov(\xi_j,\omega_j)=0$).
    \item Observable $\kappa$ conduct shifters (financial mergers/events, see Miller Weinberg (2018))
\end{itemize}
\end{frame}


\begin{frame}[shrink=25,plain]{Algorithm}
\begin{enumerate}[(1)]
\item Split the sample by markets $t$ into 70\% \textit{test} and 30\% \textit{train}.
\item On the \textit{training sample}:
\begin{enumerate}[(a)]
\item Approximate the optimal instruments $a(z_{jt}^s) = \mathbb{E}[\Delta \eta_{jt}^{(1,2)} \mid z_{jt}^s]$ as the fitted values from:
\begin{align*}
    \Delta\eta_{jt}^{1,2} = g(z_{jt}^s) + \zeta_{jt}.
\end{align*}
\item Estimate the marginal cost function, under models 1 and 2 to obtain residuals $\widehat{\omega_{jt}^{1}}$ and $\widehat{\omega_{jt}^{2}}$:
 \begin{align*}
 p_{jt} -\eta^m_{jt}= h_s(\textrm{x}_{jt},\mathrm{w}_{jt}; \theta_3) + \omega^m_{jt}.
 \end{align*}
\end{enumerate}
\item On the \textit{test sample}:
\begin{enumerate}[(a)]
\item For each candidate model, compute the value of the scalar moment:\footnotemark
\begin{align*}
 Q(\eta^m) =\left(\sum_{j,t} \hat\omega_{jt}^m\cdot \hat{g}(\symbf{z_t}) \right)^2.
\end{align*}
\item Repeat the previous step on bootstrapped samples and estimate $\hat\sigma/\sqrt{n}$ the standard error of the difference $\tilde Q(\eta^1) - \tilde Q(\eta^2)$.
\item Compute the test statistic
\begin{align*}
T = \frac{\sqrt{n} \left(Q(\eta^1) -  Q(\eta^2) \right)}{\widehat{\sigma}} \sim \mathcal{N}(0,1).
\end{align*}
\textit{Note: Steps 2(a) and 2(b) can be done in any order via non-parametric regression.}
\end{enumerate}
\end{enumerate}
\end{frame}

\begin{frame}[plain]{Limitations}
Not everything is testable:
\begin{itemize}
\item If $\Delta \eta_{jt}$ cannot be explained by $z_{jt}^s$ beyond contents of $(\mathrm{x}_j,\mathrm{w}_j)$ we have nothing
%\item Compare perfect competition to a fixed markup $p_j = a \cdot mc_j + b$
\item Flexible demand models are required to generate cross sectional variation in markups
%\begin{itemize}
%\item Discuss plain logit
%\end{itemize}
\item Beware of ``accidental'' exclusion restrictions.
\end{itemize}
\end{frame}

\begin{frame}[plain,label=advantages]{Alternate Justifications}
\begin{enumerate}
\item Model Misspecification: if one of the two models is correct, $\eta^1_{jt}-\eta_{jt}^2$ exactly corresponds to the misspecification error when using the other. \hyperlink{misspecification}{\beamerskipbutton{detail}}. 
\item Difference in Test Statistics: We can also show it maximizes the difference in GMM objectives $(Q_1 - Q_2)$. (Which is almost power for fixed $\sigma$). Also we need that either $h_s(\cdot)$ doesn't depend on $\eta^{(m)}$ or that it is linear so we can residualize on $(\textrm{x},\text{w})$.
\end{enumerate}
\end{frame}







\end{document}


\item It turns out that 2SLS analog $E[\Delta \eta_{jt} | x_t, w_t, v_t,Z_{jt}^e]=\widehat{\Delta \eta_{jt}}$ doesn't add much:
\begin{itemize}
\item Markups aren't a linear function of observables.
\item Coefficients are (probably) quite different across products.
\end{itemize}