%2multibyte Version: 5.50.0.2953 CodePage: 65001

% Use the shared preamble (preserve per-file beamer options)
\def\beamerclassoptions{[notes=show]}
\documentclass[10pt,aspectratio=169]{beamer}

% silence some Metropolis warnings
\usepackage{silence}
\WarningFilter{beamerthememetropolis}{You need to compile with XeLaTeX or LuaLaTeX}
\WarningFilter{latexfont}{Font shape}
\WarningFilter{latexfont}{Some font}

% define custom colors
\usepackage{xcolor}
\definecolor{dark gray}{HTML}{444444}
\definecolor{light gray}{HTML}{777777}
\definecolor{dark red}{HTML}{BB0000}
\definecolor{dark green}{HTML}{00BB00}
\definecolor{RoyalBlue}{cmyk}{1, 0.50, 0, 0}

% configure metropolis
\usetheme[numbering=fraction]{metropolis}
\setbeamercolor{background canvas}{bg=white}
\setbeamercolor{frametitle}{bg=dark gray}
\setbeamercolor{alerted text}{fg=dark red}
\setbeamercolor{item projected}{bg=dark red}
\setbeamercolor{local structure}{fg=dark red}
\setbeamersize{text margin left=0.5cm,text margin right=0.5cm}
\setbeamercovered{transparent=10}

% use thicker lines
\makeatletter
\setlength{\metropolis@titleseparator@linewidth}{1pt}
\setlength{\metropolis@progressonsectionpage@linewidth}{1pt}
\makeatother

% custom bullet points
\setbeamertemplate{itemize item}{\color{dark red}$\blacktriangleright$}
\setbeamertemplate{itemize subitem}{\color{dark red}$\blacktriangleright$}
\setbeamertemplate{itemize subsubitem}{\color{dark red}$\blacktriangleright$}
\newcommand{\custombullet}{{\color{dark red}$\blacktriangleright$}\hspace{0.5em}}


% imports
\usepackage[english]{babel}
\usepackage[utf8]{inputenc}
\usepackage{amsthm}
\usepackage{amssymb}
\usepackage{amsmath}
\usepackage{amsfonts}
\usepackage{mathtools}
\usepackage{mathabx}
\usepackage{stmaryrd}
\usepackage{graphicx}
\usepackage{hyperref}
\usepackage{xfrac}
\usepackage{appendixnumberbeamer}
\usepackage{tabularx}

% check and x marks
\usepackage{pifont}
\newcommand{\cmark}{{\color{dark green}\ding{51}}\hspace{0.3em}}
\newcommand{\xmark}{{\color{dark red}\ding{55}}\hspace{0.5em}}


% use classic font for math
\usepackage[T1]{fontenc} % Needed for Type1 Concrete \usepackage{concmath}
\usefonttheme{serif}
\usefonttheme{professionalfonts}
\usepackage{concmath}
\setbeamerfont{equation}{size=\tiny}



% diagrams
\usepackage{tikz}
\usetikzlibrary{decorations.pathreplacing}

% references
\usepackage[natbibapa]{apacite}
\bibliographystyle{apacite}
\renewcommand{\bibsection}{}

% use ampersands instead of "and" for text citations
\AtBeginDocument{\renewcommand{\BBAB}{\&}}

% possessive cites
\makeatletter
\patchcmd{\NAT@test}{\else \NAT@nm}{\else \NAT@nmfmt{\NAT@nm}}{}{}
\DeclareRobustCommand\citepos
  {\begingroup
   \let\NAT@nmfmt\NAT@posfmt
   \NAT@swafalse\let\NAT@ctype\z@\NAT@partrue
   \@ifstar{\NAT@fulltrue\NAT@citetp}{\NAT@fullfalse\NAT@citetp}}
\let\NAT@orig@nmfmt\NAT@nmfmt
\def\NAT@posfmt#1{\NAT@orig@nmfmt{#1's}}
\makeatother

% spaced-out lists
\newenvironment{wideitemize}{\itemize\addtolength{\itemsep}{10pt}}{\enditemize}
\newenvironment{wideenumerate}{\enumerate\addtolength{\itemsep}{10pt}}{\endenumerate}

% replace footnotes with buttons
\usepackage[absolute,overlay]{textpos}
\newcounter{beamerpausessave}
\newcommand{\always}[1]{
    \setcounter{beamerpausessave}{\value{beamerpauses}}
    \setcounter{beamerpauses}{0}
    \pause
    #1 
    \setcounter{beamerpauses}{\value{beamerpausessave}}
    \addtocounter{beamerpauses}{-1}
    \pause
}
\newcommand{\buttons}[1]{\always{
    \begin{textblock*}{\paperwidth}(0.015\textwidth, 1.022\textheight)
        \scriptsize
        #1
    \end{textblock*}
}}
\newcommand{\appendixbuttons}[1]{\always{
    \begin{textblock*}{\paperwidth}(0.015\textwidth, 1.043\textheight)
        \scriptsize
        #1
    \end{textblock*}
}}
\newcommand{\goto}[2]{\hyperlink{#1}{{\color{dark red}$\smalltriangleright$} #2}\hspace{0.5em}}
\newcommand{\goback}[2]{\hyperlink{#1}{{\color{dark red}$\smalltriangleleft$} #2}\hspace{0.5em}}

% custom appendix
\renewcommand{\appendixname}{\texorpdfstring{\translate{Appendix}}{Appendix}}

% change color of cites and URLs
\let\oldcite\cite
\let\oldcitet\citet
\let\oldcitep\citep
\let\oldcitepos\citepos
\let\oldcitetalias\citetalias
\let\oldcitepalias\citepalias
\let\oldurl\url
\def\cite#1#{\citeaux{#1}}
\def\citet#1#{\citetaux{#1}}
\def\citep#1#{\citepaux{#1}}
\def\citepos#1#{\citeposaux{#1}}
\def\citetalias#1#{\citetaliasaux{#1}}
\def\citepalias#1#{\citepaliasaux{#1}}
\def\url#1#{\urlaux{#1}}
\newcommand*\citeaux[2]{{\color{light gray}\oldcite#1{#2}}}
\newcommand*\citetaux[2]{{\color{light gray}\oldcitet#1{#2}}}
\newcommand*\citepaux[2]{{\color{light gray}\oldcitep#1{#2}}}
\newcommand*\urlaux[2]{{\color{light gray}\oldurl#1{#2}}}
\newcommand*\citeposaux[2]{{\color{light gray}\oldcitepos#1{#2}}}
\newcommand*\citetaliasaux[2]{{\color{light gray}\oldcitetalias#1{#2}}}
\newcommand*\citepaliasaux[2]{{\color{light gray}\oldcitepalias#1{#2}}}

% custom math commands
\DeclareMathOperator*{\argmax}{argmax}
\DeclareMathOperator*{\argmin}{argmin}
\renewcommand{\Pr}{\mathbb{P}}
\newcommand{\E}{\mathbb{E}}
\newcommand{\Var}{\mathbb{V}}
\newcommand{\Cov}{\mathbb{C}}
\newcommand{\overbar}[1]{\mkern 1.5mu\overline{\mkern-1.5mu#1\mkern-1.5mu}\mkern 1.5mu}
\newcommand{\abs}[1]{\lvert#1\rvert}
\newcommand{\norm}[1]{\lVert#1\rVert}

% tables
\usepackage{booktabs}
\usepackage{colortbl}
\usepackage{multirow}
\usepackage{makecell}
\arrayrulecolor{dark red}

% custom date
\usepackage{datetime}
\newdateformat{monthyeardate}{\monthname[\THEMONTH] \THEYEAR}

% fix pauses with graphics
\usepackage{../resources/fixpauseincludegraphics}


% Local slide helpers (kept here if used in this file)
\newenvironment{stepenumerate}{\begin{enumerate}[<+->]}{\end{enumerate}}
\newenvironment{stepitemize}{\begin{itemize}[<+->]}{\end{itemize} }
\newenvironment{stepenumeratewithalert}{\begin{enumerate}[<+-| alert@+>]}{\end{enumerate}}
\newenvironment{stepitemizewithalert}{\begin{itemize}[<+-| alert@+>]}{\end{itemize} }
%\usetheme{Madrid}
%\input{tcilatex}
\begin{document}

\title[Lecture 10]{Lecture 8:\ Empirical Two Period Models}
\author{Kate Ho }
\date[11/13]{ November 2016}
\maketitle

\section{Two Period Entry Models}

%TCIMACRO{\TeXButton{BeginFrame}{\begin{frame}}}%
%BeginExpansion
\begin{frame}%
%EndExpansion
\frametitle{Two Period Entry Models}

Begin with entry models and then (next class) talk about other forms of
investments. Outline for entry models:

\begin{itemize}
\item Introduction and overview

\item Models with identical firms

\item Models with firm heterogeneity in fixed costs

\item Entry models with heterogeneity in continuation values:\ asymmetric
information in simple differentiation models; perfect information in
sequential models.
\end{itemize}

%TCIMACRO{\TeXButton{EndFrame}{\end{frame}}}%
%BeginExpansion
\end{frame}%
%EndExpansion

%TCIMACRO{\TeXButton{BeginFrame}{\begin{frame}}}%
%BeginExpansion
\begin{frame}%
%EndExpansion

\frametitle{Introduction}

In two-period entry models firms make an entry choice in period 1, and then
compete (in prices or quantities) in period 2 conditional on first period
choices.

Solution concept is \textquotedblleft subgame perfection\textquotedblright\
so - if there's complete information and a finite horizon - we solve the
game backwards

\begin{itemize}
\item Solve for period 2 Nash quantities (or prices) resulting from every
set of period 1 choices

\item Assume the period 2 choices are (always) unique. Compute resulting
period 2 profits and net cash flow (profits minus sunk cost on entry) for
each active firm, for each possible period 1 choice

\item Find a Nash equilibrium in the entry (period one) choices.
\end{itemize}

%TCIMACRO{\TeXButton{EndFrame}{\end{frame}}}%
%BeginExpansion
\end{frame}%
%EndExpansion

%TCIMACRO{\TeXButton{BeginFrame}{\begin{frame}}}%
%BeginExpansion
\begin{frame}%
%EndExpansion

\frametitle{Introduction, cntd.}

These models have been used extensively in the theoretical literature to
develop intuition on what can happen once we endogenize entry.

\bigskip

They have been applied empirically to explain determinants of
cross-sectional differences in market structure. They are not "structural
models":\ we cannot use them to estimate primitives and do counterfactuals.
They are a reduced form way of summarizing the data on cross-sectional
differences that can be quite helpful.

\bigskip

They can be misleading because they miss the dynamics of a true sequential
game. E.g.:\ if firms have been active for a long time then differences in
market structure may be more related to differences in market conditions
many years ago than to differences today.

\bigskip

But working through the simple models does give us an idea of what a value
function might have to look like to match cross-sectional differences (and
therefore what markups would have to look like to do this). I.e. these
models can be suggestive about the likely strengths of various entry
incentives.

%TCIMACRO{\TeXButton{EndFrame}{\end{frame}}}%
%BeginExpansion
\end{frame}%
%EndExpansion

%TCIMACRO{\TeXButton{BeginFrame}{\begin{frame}}}%
%BeginExpansion
\begin{frame}%
%EndExpansion

\frametitle{Entry Models with Identical Firms.}

\begin{itemize}
\item In these models a set of identical firms make a period 1 choice of
whether to enter a market. In period 2 entrants compete according to an
oligopoly game such as Cournot.

\item Because firms are identical, after entry they split the market:\ each
of $N$ entering firms receives $1/N$ of total market demand.

\item When these models consider product differentiation, demand function is
\textquotedblleft symmetric\textquotedblright\ (e.g. logits with identical $%
\delta $) so market shares remain equal.

\item E.g. with Cournot post-entry competition. Value function is $V(N)$,
depending only on the number of firms (not their identity) because of
symmetry.

\item Equilibrium number of firms is the largest $N$ such that $V(N)>0$.

\item Mankiw and Whinston (RAND,86) provide a good introduction to this
class of models and consider their implications for \textquotedblleft
welfare\textquotedblright .
\end{itemize}

%TCIMACRO{\TeXButton{EndFrame}{\end{frame}}}%
%BeginExpansion
\end{frame}%
%EndExpansion

%TCIMACRO{\TeXButton{BeginFrame}{\begin{frame}}}%
%BeginExpansion
\begin{frame}%
%EndExpansion

\frametitle{Entry Models with Identical Firms, cntd}

\textit{Exercise}:

Consider the Cournot second period competition with the linear inverse
demand function $p=a-bQ$ and the cost function $C(q)=cq+F$. Derive the
free-entry equilibrium number of firms, $N^{e}$ . Now consider the problem
of a social planner who can choose the number of firms, but cannot set
prices or quantities. Derive this social planner's optimal number of firms, $%
N^{\ast }$. Show that $N^{e}=(N^{\ast }+1)^{3/2}-1$. Note that $%
N^{e}>N^{\ast }$ , which is a general property of this class of models.

%TCIMACRO{\TeXButton{EndFrame}{\end{frame}}}%
%BeginExpansion
\end{frame}%
%EndExpansion

%TCIMACRO{\TeXButton{BeginFrame}{\begin{frame}}}%
%BeginExpansion
\begin{frame}%
%EndExpansion

\frametitle{Estimating Value
Function Parameters from Cross-Sectional Data on
 Entry with Identical Firms}

\begin{itemize}
\item These symmetric two-period entry models lead to a natural estimation
strategy

\item Idea is revealed preference: if a firm operates in a market it must
have a positive \textquotedblleft entry value\textquotedblright ; if it
doesn't operate then must not

\item Use some kind of discrete choice framework to infer how different
variables affect the value function.

\item Variables of interest might include demand and cost shifters and/or
endogenous variables e.g. the number of competitors.

\item Data:\ typically a cross-section of markets where we observe the
number of competitors$\ N$ and some market-specific profit-shifters, $x$
(e.g. market size, other demand measures, cost measures) but no firm-level
data. The model finds $N$ as a function of $x$ and the realizations of an
error process.
\end{itemize}

%TCIMACRO{\TeXButton{EndFrame}{\end{frame}}}%
%BeginExpansion
\end{frame}%
%EndExpansion

%TCIMACRO{\TeXButton{BeginFrame}{\begin{frame}}}%
%BeginExpansion
\begin{frame}%
%EndExpansion

\frametitle{Bresnahan and Reiss (1987,1990,1991)}

\begin{itemize}
\item Look at retail and professional firms in small isolated markets:\ so
small that the number of establishments is also small.

\item They carefully pick towns in isolated parts of the country (e.g. rural
Nevada), to reduce interference from nearby markets

\item Cross-sectional observations on town characteristics (including
population) and number of establishments.

\item First part of model uses revealed preference idea to find value of
entering a market $V(N)$. Second part uses an insight about the number of
customers and market conditions needed to support each firm to make
inferences about type of competition in the market.
\end{itemize}

\bigskip

This was a very innovative set of papers. (However there are interesting
data issues that the authors ignore, e.g. definition of an establishment.
For example, does a multi-doctor practice at one location count as one
establishment or two? I.e. there are questions of scale and scope that are
difficult to address without both a fuller model and more data.)

%TCIMACRO{\TeXButton{EndFrame}{\end{frame}}}%
%BeginExpansion
\end{frame}%
%EndExpansion

%TCIMACRO{\TeXButton{BeginFrame}{\begin{frame}}}%
%BeginExpansion
\begin{frame}%
%EndExpansion

\frametitle{The Model (a simplified version)}

Define the value of entering market $i$ as 
\begin{equation}
EV(N_{i})=VC(N\,_{i},M_{i},x_{i},\theta _{2})-FC_{i}  \label{entry1}
\end{equation}

\begin{itemize}
\item where $N_{i}$ is the observed number of firms in market $i$, $M_{i}$
is the size of market $i$, $VC(.)$ is the continuation value (so it could
include indicators of future profitability, like expected growth rates, as
well as indicators of current profits), $x_{i}$ is a vector of profit
shifters and $\theta _{2}$ is a vector of parameters to be estimated.

\item Fixed costs, $FC_{i}$, are common to each firm in the market and are
know to the firms but not observed by the econometrician.

\item Both fixed costs and continuation values are market, not firm,
specific.
\end{itemize}

%TCIMACRO{\TeXButton{EndFrame}{\end{frame}}}%
%BeginExpansion
\end{frame}%
%EndExpansion

%TCIMACRO{\TeXButton{BeginFrame}{\begin{frame}}}%
%BeginExpansion
\begin{frame}%
%EndExpansion

\frametitle{The Model, cntd.}

Nash Equilibrium implies that the equilibrium (observed) number of firms in
a market satisfies 
\begin{equation}
EV(N_{i})>0>EV(N_{i}+1)  \label{entry2}
\end{equation}

or, plugging in the formula for value, 
\begin{equation}
VC(N\,_{i},M_{i},x_{i},\theta )>FC_{i}>VC(N\,_{i}+1,M_{i},x_{i},\theta )
\label{entry3}
\end{equation}

\begin{itemize}
\item These fixed costs are assumed to be identical for all potential
entrants in a given market, but may differ over markets.

\item The probability of observing $N_{i}$ firms therefore depends on the
probability that fixed costs in this market fall between some upper and
lower bound.
\end{itemize}

%TCIMACRO{\TeXButton{EndFrame}{\end{frame}}}%
%BeginExpansion
\end{frame}%
%EndExpansion

%TCIMACRO{\TeXButton{BeginFrame}{\begin{frame}}}%
%BeginExpansion
\begin{frame}%
%EndExpansion

\frametitle{The Model, cntd.}

\begin{itemize}
\item To calculate this probability, we make a parametric assumption on the
distribution of fixed costs. If $FC\sim \Phi (.;\theta _{2})$ then the
likelihood of $N_{i}$ firms is 
\begin{eqnarray}
\Pr (\text{Observe }N_{i}\text{ firms}) &=&\Phi
(VC(N_{i},M\,_{i},x_{i},\theta _{1});\theta _{2})  \label{entry4} \\
&&-\Phi (VC(N_{i}+1,M\,_{i},x_{i},\theta _{1});\theta _{2})
\end{eqnarray}
\end{itemize}

and one can estimate $\theta $ by MLE. If the fixed costs distribute (log)
normally, then this likelihood is an \textquotedblleft ordered
probit.\textquotedblright\ 

%TCIMACRO{\TeXButton{EndFrame}{\end{frame}}}%
%BeginExpansion
\end{frame}%
%EndExpansion

%TCIMACRO{\TeXButton{BeginFrame}{\begin{frame}}}%
%BeginExpansion
\begin{frame}%
%EndExpansion

\frametitle{The Model:\ Second Step}

\begin{itemize}
\item Then B\&R use the parameters to make inferences about the
\textquotedblleft nature of competition\textquotedblright .

\item Their argument concerns the continuation value needed to support the
firm and therefore the market size and market characteristics needed to
support it.

\item \textquotedblleft Benchmark\textquotedblright\ case:\ assume prices
and demand are unchanged after entry. Then $VC(N)$ will decrease
proportionately with $N$.

\begin{itemize}
\item e.g. $VC(2)=\frac{1}{2}VC(1)$.

\item Approximates a homogeneous goods market where entry doesn't drive
prices down.
\end{itemize}
\end{itemize}

%TCIMACRO{\TeXButton{EndFrame}{\end{frame}}}%
%BeginExpansion
\end{frame}%
%EndExpansion

%TCIMACRO{\TeXButton{BeginFrame}{\begin{frame}}}%
%BeginExpansion
\begin{frame}%
%EndExpansion

\frametitle{The Model:\ Second Step, cntd.}

\begin{itemize}
\item Then define the "entry thresholds"\ $M_{N}^{\ast }$ as the population
size that would induce $N$ firms to enter (but not $N+1$ firms)

\item In reality the entry threshold would change with the value of fixed
cost in the market. They remove this issue by defining $M_{N}^{\ast }$ =
entry threshold if fixed cost held at the mean value everywhere (if all
markets have same FC)

\item Then the "per firm threshold" (the population needed to support 1 firm
in a market containing $N$ firms)\ is%
\begin{equation*}
m_{N}=\frac{M_{N}^{\ast }}{N}.
\end{equation*}
\end{itemize}

%TCIMACRO{\TeXButton{EndFrame}{\end{frame}}}%
%BeginExpansion
\end{frame}%
%EndExpansion

%TCIMACRO{\TeXButton{BeginFrame}{\begin{frame}}}%
%BeginExpansion
\begin{frame}%
%EndExpansion

\frametitle{The Model: Second Step, cntd.}

\begin{itemize}
\item In the benchmark case (e.g. homogeneous goods and no price
competition) $V(N)$ decreases linearly as $N$ increases, so the per-firm
thresholds are constant in $N$: 
\begin{equation*}
\frac{m_{N+1}}{m_{N}}=1.
\end{equation*}

\item e.g. if price and demand are unchanged by entry, we need to double the
total market size to support 2 firms rather than 1.

\item But if prices are driven down by entry, $VC(N)$ should decrease more
rapidly than $N$ increases (e.g. $VC(2)<\frac{1}{2}VC(1)$);

\item In that case, to make up a given fixed cost we would have to have a
larger market (remember the firms are symmetric, so ex poste they all make
equal profits). So $m_{N+1}>m_{N}$ or 
\begin{equation*}
\frac{m_{N+1}}{m_{N}}>1.
\end{equation*}

\item i.e. now we need to more-than-double the market size to support 2
firms rather than 1.
\end{itemize}

%TCIMACRO{\TeXButton{EndFrame}{\end{frame}}}%
%BeginExpansion
\end{frame}%
%EndExpansion

%TCIMACRO{\TeXButton{BeginFrame}{\begin{frame}}}%
%BeginExpansion
\begin{frame}%
%EndExpansion

\frametitle{The Model:\ Notes}

In this model just one unobservable (the market specific level of $FC$)
generates all the outcomes for $N_{i}$, and there are no differences between
firms. We know this is unrealistic. Also there may be reasons why $\frac{%
m_{N+1}}{m_{N}}=1$ or even lower, even in cases where prices are driven down
by entry. For example:\ 

\begin{itemize}
\item Fixed costs that varied across entrants. E.g. if fixed costs increased
as $N$ increased then this would increase $\frac{m_{N+1}}{m_{N}}$; B\&R\
allow for this later in the paper.

\item In more complicated cases of fixed cost heterogeneity we would need a
more detailed model:\ to specify the distribution of fixed costs for each
potential entrant, to determine a unique number of entrants for each
possible realization of the fixed cost vector (say let the lowest fixed cost
entrant entered first).
\end{itemize}

%TCIMACRO{\TeXButton{EndFrame}{\end{frame}}}%
%BeginExpansion
\end{frame}%
%EndExpansion

%TCIMACRO{\TeXButton{BeginFrame}{\begin{frame}}}%
%BeginExpansion
\begin{frame}%
%EndExpansion

\frametitle{The Model:\ Notes, cntd.}

Other reasons why we could have $\frac{m_{N+1}}{m_{N}}=1$ or even lower,
even in cases where prices are driven down by entry:

\begin{itemize}
\item Continuation values which differ (e.g. due to product
differentiation). Now entry of a new firm may expand the overall size of the
market making it difficult to put a sign on the effect. For example, if the
2 entrants' products are not substitutes, then entry has no effect on
quantity or price and it could be that $VC(2)=VC(1)$.

\item With price competition and product differentiation, it is difficult to
make any prediction on how the equilibrium changes with the number of
entrants; see Shaked and Sutton(.) for a discussion.
\end{itemize}

We will consider these cases later in the class.

%TCIMACRO{\TeXButton{EndFrame}{\end{frame}}}%
%BeginExpansion
\end{frame}%
%EndExpansion

%TCIMACRO{\TeXButton{BeginFrame}{\begin{frame}}}%
%BeginExpansion
\begin{frame}%
%EndExpansion

\frametitle{The Model:\ Results}

Estimated ratios of per-firm thresholds. B \& R interpret the ratios greater
than one as evidence of prices declining in N.

\begin{center}
\begin{tabular}{|l|l|l|l|l|}
\hline
Profession & $m\,_{2}/m_{1}$ & $m\,_{3}/m_{2}$ & $m\,_{4}/m_{3}$ & $%
m\,_{5}/m_{4}$ \\ \hline
Doctors & 1.98 & 1.10 & 1.00 & 0.95 \\ \hline
Dentists & 1.78 & 0.79 & 0.97 & 0.94 \\ \hline
Plumbers & 1.06 & 1.00 & 1.02 & 0.96 \\ \hline
Tire Dealers & 1.81 & 1.28 & 1.04 & 1.03 \\ \hline
\end{tabular}
\end{center}

(Per Firm Entry Thresholds from Bresnahan and Reiss, 1991 Table 5\bigskip )

\begin{itemize}
\item Prices seem to decline a lot when moving from one to two doctors, tire
dealers or dentists.

\item But further increases in $N$ do not seem to increase competition much.

\item Consistent with old jokes, plumbers' prices never fall.
\end{itemize}

%TCIMACRO{\TeXButton{EndFrame}{\end{frame}}}%
%BeginExpansion
\end{frame}%
%EndExpansion

%TCIMACRO{\TeXButton{BeginFrame}{\begin{frame}}}%
%BeginExpansion
\begin{frame}%
%EndExpansion

\frametitle{The Model:\ Results cntd.}

The caveats above may of course apply.

\bigskip

But there is good intuition behind the answers and none of our caveats
explains why so many of the estimated threshold ratios are close to exactly
one.

\bigskip

B \& R also look at the observed prices of tire dealers in a simple
regression context. These prices do seem to fall with the first few entrants
and then level out. This is consistent with the estimated entry thresholds.
They also note the interesting fact that the prices in these small towns
appear to level out at a much higher level than is observed at large,
big-city tire dealers.

%TCIMACRO{\TeXButton{EndFrame}{\end{frame}}}%
%BeginExpansion
\end{frame}%
%EndExpansion

%TCIMACRO{\TeXButton{BeginFrame}{\begin{frame}}}%
%BeginExpansion
\begin{frame}%
%EndExpansion

\frametitle{Extensions to BR.}

We can extend the simple framework of the last section in several ways, and
still maintain the the ordered probit estimation method. These include

\begin{itemize}
\item Let the size of the market, $M_{i}$ depend on some data and parameters
(e.g. distance to nearby population centers, own-town population, average
income ...). BR do this.

\item Introduce some heterogeneity in the mean of the fixed costs
distribution. BR do this too (they allow the mean fixed costs of the
\textquotedblleft first entrant\textquotedblright\ to differ from the mean
fixed costs of the \textquotedblleft second entrant\textquotedblright ).

\item Add price and or quantity data:\ this is the next paper to discuss.
\end{itemize}

%TCIMACRO{\TeXButton{EndFrame}{\end{frame}}}%
%BeginExpansion
\end{frame}%
%EndExpansion

%TCIMACRO{\TeXButton{BeginFrame}{\begin{frame}}}%
%BeginExpansion
\begin{frame}%
%EndExpansion

\frametitle{Adding Price and Quantity Data}

\begin{itemize}
\item B\&R's data don't include output quantities (also very little on
prices), so they use only observed entry data to learn about $VC(N)$

\item Berry-Waldfogel (RAND 1999) consider entry into the radio industry,
where both price and quantity data are available.

\item They assume a simple parametric relationship between continuation
values $VC(N)$ and current profits (largely implicit in B\&R anyway) and use
prices and quantities to estimate $VC(N)$.

\item Entry data are only needed to estimate distribution of fixed costs.
\end{itemize}

(I.e. they make assumptions that enable them to estimate the parameters of
the actual profit function, a function of prices and quantities, not just a
reduced form relationship between the continuation value and
demographics/number of entrants as in BR.)

%TCIMACRO{\TeXButton{EndFrame}{\end{frame}}}%
%BeginExpansion
\end{frame}%
%EndExpansion

%TCIMACRO{\TeXButton{BeginFrame}{\begin{frame}}}%
%BeginExpansion
\begin{frame}%
%EndExpansion

\frametitle{Berry and Waldfogel (1999):\ Overview}

A three-equation model:

\begin{enumerate}
\item Slope of listeners as a function of $N.$ Stations ``produce" listeners,
who make a free choice over stations. They use a nested logit. Each station
produces an idiosyncratic benefit, so listening increases in $N$; but there
is still a business stealing effect generated by entry.The logit functional
form determines the effect of $N$ on total listening (so one should think
hard about its implications).

\item Revenues as a function of number of listeners (and therefore of $N$).
Stations sell listeners to advertisers at some price per listener-hour.
Advertisers' demand is downward sloping with a simple constant elasticity
functional form.

\item Entry equation. Free entry with identical firms:\ so given estimates
of revenue as a function of $N$ from equations 1-2, the entry equation is
the same as in Bresnahan and Reiss (equation (\ref{entry2})). If the
(identical) fixed costs distribute (log) normally across markets, the entry
equation is an ordered probit.
\end{enumerate}

%TCIMACRO{\TeXButton{EndFrame}{\end{frame}}}%
%BeginExpansion
\end{frame}%
%EndExpansion

%TCIMACRO{\TeXButton{BeginFrame}{\begin{frame}}}%
%BeginExpansion
\begin{frame}%
%EndExpansion

\frametitle{Berry and Waldfogel, cntd.}

\begin{itemize}
\item Data:\ a cross section of 135 metro areas

\item Results are used to look at relationship between PS and entry.

\item Idea:\ as in Mankiw and Whinston (1986), there is a \textquotedblleft\
business stealing\textquotedblright\ effect of new entry on the profits of
existing firms

\item Empirical results show a lot of business stealing by new stations,
implying a large PS loss from free entry (approx 40\% of industry revenue).

\item Implies a large potential gain from cartelization
\end{itemize}

%TCIMACRO{\TeXButton{EndFrame}{\end{frame}}}%
%BeginExpansion
\end{frame}%
%EndExpansion

%TCIMACRO{\TeXButton{BeginFrame}{\begin{frame}}}%
%BeginExpansion
\begin{frame}%
%EndExpansion

\frametitle{Berry and Waldfogel, cntd.}

These results are interesting, but problematic for several reasons in
addition to the general problems with the two period models.

\begin{itemize}
\item The results consider only PS. For a social planner CS matters too.

\item Authors partially address this by trying to calculate externalities to
listeners as a result of entry.

\item But we've already discussed problems of estimating CS from logit
estimates, plus here there's no mechanism for advertising to affect
consumers and not enough flexibility to allow product diversity to affect
utility.

\item Note:\ the FCC established new, more liberal regulations on ownership
in the radio industry from 1996 and there's debate on further
liberalization. It would be interesting to investigate likely welfare
effects.

\item Waldfogel (2003) and Sweeting (2012) have done relevant work here.
\end{itemize}

%TCIMACRO{\TeXButton{EndFrame}{\end{frame}}}%
%BeginExpansion
\end{frame}%
%EndExpansion

%TCIMACRO{\TeXButton{BeginFrame}{\begin{frame}}}%
%BeginExpansion
\begin{frame}%
%EndExpansion

\frametitle{Firm Heterogeneity}

\begin{itemize}
\item Two different ways of allowing for firm heterogeneity have been
introduced into the entry models.

\item The first allows for fixed costs:\ this was discussed in B\&R (who
have some early results) and was extended by Berry (1992) who showed how to
use simulation to ameliorate the computational problems with using the model.

\item The second is heterogeneity in variable costs or in continuation
values. There have been a number of ways recently proposed to do this; we
will focus on Seim (2005), Mazzeo(2004) and Toivanen and Waterson (1999).
\end{itemize}

%TCIMACRO{\TeXButton{EndFrame}{\end{frame}}}%
%BeginExpansion
\end{frame}%
%EndExpansion

%TCIMACRO{\TeXButton{BeginFrame}{\begin{frame}}}%
%BeginExpansion
\begin{frame}%
%EndExpansion

\frametitle{Heterogeneity in Fixed Costs:\ Berry (1992)}

\begin{itemize}
\item Berry considers entry of airlines operating on city-pair routes soon
after deregulation of the U.S. airline industry.

\item Keeps the assumption that continuation values $VC(N)$ are the same
across firms

\item But lets $FC$ vary across firms. Entry values are given by 
\begin{equation}
EV(N,x_{m},z_{f},\varepsilon _{mf},\theta )=VC(N,x_{m},\theta
)-FC(z_{f},\varepsilon _{mf},\theta )  \label{entry5}
\end{equation}%
where $x_{m}$ are market-specific variables, $z_{f}$ are firm specific
variables, $\varepsilon $ is a firm/market specific profit shock and $\theta 
$ is a vector of parameters to be estimated. Note the restriction that
firm-specific variables do not enter continuation values (as before). Once
we relax this assumption we will require both a different \textquotedblleft
notion\textquotedblright\ of equilibrium, and a different estimation routine.
\end{itemize}

%TCIMACRO{\TeXButton{EndFrame}{\end{frame}}}%
%BeginExpansion
\end{frame}%
%EndExpansion

%TCIMACRO{\TeXButton{BeginFrame}{\begin{frame}}}%
%BeginExpansion
\begin{frame}%
%EndExpansion

\frametitle{Berry (1992), cntd.}

To parametrize even further, assume that:\ \ 
\begin{equation}
EV_{mf}(N)=x_{m}\beta +\delta \ln (N)-z_{mf}\alpha -\varepsilon _{mf}
\label{entry6}
\end{equation}

with $\varepsilon $ modeled as having a market specific component, which
allows for correlation in unobservables across firms within the market, 
\begin{equation*}
\varepsilon _{mf}=\rho u_{0m}+\sqrt{1-\rho ^{2}}u_{mf}
\end{equation*}

(Note the constraint on the variances to set the variance of $\varepsilon $
equal to one; similar to any discrete choice model.)

%TCIMACRO{\TeXButton{EndFrame}{\end{frame}}}%
%BeginExpansion
\end{frame}%
%EndExpansion

%TCIMACRO{\TeXButton{BeginFrame}{\begin{frame}}}%
%BeginExpansion
\begin{frame}%
%EndExpansion

\frametitle{Berry (1992), cntd.}

\begin{itemize}
\item The $\varepsilon $'s are known to the firms (but not the
econometrician).

\item This is a full information Nash equilibrium (all firms know each
other's entry costs).

\item The probability of an $N$ firm equilibrium is not an ordered probit
because the unobservables are now a vector of dimension equal to the number
of potential firms.

\item Deriving the likelihood requires us to

\begin{itemize}
\item Find the $\varepsilon $'s that generate a particular outcome variable
(this will be the number of entrants),

\item Find the probability of observing these $\varepsilon $'s (i.e. the
probability of the observed outcome) as a function of the parameters.
\end{itemize}
\end{itemize}

%TCIMACRO{\TeXButton{EndFrame}{\end{frame}}}%
%BeginExpansion
\end{frame}%
%EndExpansion

%TCIMACRO{\TeXButton{BeginFrame}{\begin{frame}}}%
%BeginExpansion
\begin{frame}%
%EndExpansion

\frametitle{Berry (1992), cntd.}

\begin{itemize}
\item In these more general discrete entry models, it is not clear that an
equilibrium exists or is unique.

\item But Berry shows that if the continuation values are symmetric (depend
on $N$ but not $z_{mf}$ ) and there is a finite number of potential
entrants, then there is a simple constructive proof of an equilibrium which
detemines a unique $N$ (though not a unique set of identities to the
entrants).

\item \textit{Proof} of unique N (sketch): Order firms in decreasing entry
values, let entry occur until last profitable entry. Call this last firm $%
N^{\ast }$. This allocation of firms to \textquotedblleft
in\textquotedblright\ and \textquotedblleft out\textquotedblright\ is an
equilibrium. Can't have fewer firms in equilibrium, because firm $N^{\ast }$
would then enter. Can't have more because $N^{\ast }+1$'th firm won't be
profitable in an $N^{\ast }+1$ equilibrium.
\end{itemize}

%TCIMACRO{\TeXButton{EndFrame}{\end{frame}}}%
%BeginExpansion
\end{frame}%
%EndExpansion

%TCIMACRO{\TeXButton{BeginFrame}{\begin{frame}}}%
%BeginExpansion
\begin{frame}%
%EndExpansion

\frametitle{Berry (1992), cntd.}

{\small (See the paper for details on existence and uniqueness. Existence
comes from assuming that firm profits decrease in rivals' entry decisions
and that we can rank firms in order of profitability. Uniqueness comes from
firm heterogeneity entering only through the fixed portion of profits, ie
through }$FC${\small . Ranking by decreasing profitability therefore implies
a ranking by increasing }$FC${\small \ and the paper contains a proof that
this is enough. (If not it could be that, if }$j${\small \ entered then }$%
j+1 ${\small \ could enter too but if }$j+1${\small \ entered first then }$j~
${\small could not.))}

%TCIMACRO{\TeXButton{EndFrame}{\end{frame}}}%
%BeginExpansion
\end{frame}%
%EndExpansion

%TCIMACRO{\TeXButton{BeginFrame}{\begin{frame}}}%
%BeginExpansion
\begin{frame}%
%EndExpansion

\frametitle{Berry (1992), cntd.}

\begin{itemize}
\item While the number of firms is unique, the identities of the entering
firms are not.

\item It can happen that in equilibrium either firm A or firm B could enter,
but not both. (This happens when both firms would make profits in a $N^{\ast
}-1$ equilibrium but neither would in a $N^{\ast }+1$ equilibrium.)

\item So the model does not identify the likelihood function for the
identities of the entering firms without some further assumption (e.g. on
the order of entry.)

\item We therefore lose what information there might be in who enters, and
just keep the information on the number of firms that enter.
\end{itemize}

%TCIMACRO{\TeXButton{EndFrame}{\end{frame}}}%
%BeginExpansion
\end{frame}%
%EndExpansion

%TCIMACRO{\TeXButton{BeginFrame}{\begin{frame}}}%
%BeginExpansion
\begin{frame}%
%EndExpansion

\frametitle{Berry (1992), cntd.}

The likelihood (the probability of drawing the $\varepsilon $'s that lead to 
$N$ entrants) is hard to calculate because the region of the $\varepsilon $
space that leads to an $N$-firm equilibrium is hard to describe. But Berry
notes that it's easy to use simulation methods to solve the problem.

\begin{itemize}
\item Begin by taking $S$ draws on the underlying random variables $u$ in
equations (6)-(7).

\item For each guess of $\theta =(\alpha ,\beta ,\rho ,\delta )$ and each
draw of the $u$'s, construct the (unique) equilibrium number of firms, $\hat{%
N}(u_{s},\theta )$, via the constructive method of the equilibrium proof
just given. Holding $\theta $ fixed, integrate over the draws on $u$ to find
the probability of $N$ entrants for any $N$. (Show a picture in two
dimensional space.)

\item Iterate over guesses for $\theta $ to maximize the probability of the
observed $N$ (simulated MLE) or to fit the observed to the predicted $N$
(method of simulated moments) where $\hat{N}=E(N)$ is a continuous variable.
Berry does the latter.
\end{itemize}

%TCIMACRO{\TeXButton{EndFrame}{\end{frame}}}%
%BeginExpansion
\end{frame}%
%EndExpansion

%TCIMACRO{\TeXButton{BeginFrame}{\begin{frame}}}%
%BeginExpansion
\begin{frame}%
%EndExpansion

\frametitle{Berry (1992), cntd.}

\begin{itemize}
\item Note that for all this to work, we need to hold the random draws
constant as we move the parameter vector, just as we always do when we use
other simulation estimators (McFadden, 1986, Pakes and Pollard, 1986).

\item Simulation has wide applicability in estimating game-theoretic models
where the controls take on discrete values, because

\begin{itemize}
\item Even if we know a characteristic of the equilibrium (e.g. $N$) which
is uniquely determined by the (observed and unobserved) characteristics of
firms and the parameters, often the set of unobservables that lead to that
characteristic has complicated boundaries and it's hard to integrate over
that set.

\item But it's easy to \textit{simulate} equilibrium realizations for this
characteristic conditional on parameter values and base our estimation
method on that simulation.
\end{itemize}

\item (The empirical application is to the importance of airline hubs in
early deregulated airline industry. The data are a cross-section by
city-pair market. But the application clearly isn't the focus here.)
\end{itemize}

%TCIMACRO{\TeXButton{EndFrame}{\end{frame}}}%
%BeginExpansion
\end{frame}%
%EndExpansion

%TCIMACRO{\TeXButton{BeginFrame}{\begin{frame}}}%
%BeginExpansion
\begin{frame}%
%EndExpansion

\frametitle{A Start on Variable Continuation Values:\ Seim (2006)}

\begin{itemize}
\item Seim models geographic differentiation in the video retail market

\item $N^{\ast }$ firms decide whether to enter a given market (indexed by $%
m $); those that enter choose their location from a finite set of $L$
locations.

\item Each firm makes its choices based on the expected post entry value.

\item Seim assumes asymmetric information (firms know their own unobservable 
$\eta _{i}^{a}$ but not other firms' $\eta _{i}^{-a}$). This makes verifying
and solving for equilibria much easier than in the Berry case.

\begin{itemize}
\item Existence follows easily from Brouwer's fixed point theorem

\item The Bayesian equilibrium concept outlined below makes solving the
model much easier.
\end{itemize}

\item So we can include more product characteristics in the model.

\item Application: video rental stores. Ideal for this type of model because
the good is homogeneous, non-storable and relatively inexpensive and the
main aspect of differentiation comes from geographic location.
\end{itemize}

%TCIMACRO{\TeXButton{EndFrame}{\end{frame}}}%
%BeginExpansion
\end{frame}%
%EndExpansion

%TCIMACRO{\TeXButton{BeginFrame}{\begin{frame}}}%
%BeginExpansion
\begin{frame}%
%EndExpansion

\frametitle{Seim\ (2006), cntd.}

Post entry value in location $i$ for firm $a$ is 
\begin{eqnarray*}
v_{i}^{a} &=&\xi ^{m}+x_{i}^{m}\beta +n_{i}^{m}\theta +\sum_{j\in
d1i,j}x_{j}^{m}\beta _{1}+\sum_{j\in d2i,j}x_{j}^{m}\beta _{2}+ \\
&&\sum_{j\in d1i,j}n_{1,j}^{m}\theta _{1}+\sum_{j\in d2i,j}n_{2,j}^{m}\theta
_{2}+\eta _{i}^{a} \\
&=&\tilde{v}_{i}^{a}+\eta _{i}^{a}
\end{eqnarray*}

where

\begin{itemize}
\item $d1i,j$ is an indicator for regions adjacent to location $i$ (one to
three miles from), and $d2i,j$ is an indicator for locations somewhat
farther away from $i$ (three to ten miles from location)

\item $n_{i}^{m}$ is the number of entering firms in the immediate
\textquotedblleft tract\textquotedblright ,

\item $n_{1,j}^{m}$ is the number of firms within the distance defined from $%
d1i,j$ etc.

\item $x_{i}^{m}$ and $x_{j}^{m}$ are location characteristics

\item $\eta _{i}^{a}$ is the only difference between firms:\ unsystematic,
firm-specific characteristics.
\end{itemize}

%TCIMACRO{\TeXButton{EndFrame}{\end{frame}}}%
%BeginExpansion
\end{frame}%
%EndExpansion

%TCIMACRO{\TeXButton{BeginFrame}{\begin{frame}}}%
%BeginExpansion
\begin{frame}%
%EndExpansion

\frametitle{Seim (2006), cntd.}

\begin{itemize}
\item The mean profitability from not entering is normalized to zero (this
is a \textquotedblleft linear\textquotedblright\ discrete choice model so we
can only define things up to an affine transformation).

\item Unobservables are nested logit:\ the outer nest is \textquotedblleft
in\textquotedblright\ or \textquotedblleft out\textquotedblright , inner
nest is the choice of location.

\item (This is the same as pure logit on the inner nest choices. But there's
an extra unobservable with the same value for all the "in" choices, so they
have a correlation that isn't shared by the "out" choice.)

\item The model determines both how many firms enter, say $\hat{N}$, and a
probability distribution for their locations.

\item We will focus on the equilibrium probabilities of firms entering at
different locations conditional on $\hat{N}$. (We won't run through the
section that's more similar to previous entry models:\ given these
probabilities, use revealed preference and fit observed to predicted $N$.)
\end{itemize}

%TCIMACRO{\TeXButton{EndFrame}{\end{frame}}}%
%BeginExpansion
\end{frame}%
%EndExpansion

%TCIMACRO{\TeXButton{BeginFrame}{\begin{frame}}}%
%BeginExpansion
\begin{frame}%
%EndExpansion

\frametitle{Seim (2006), cntd.}

Note that:

\begin{itemize}
\item You might think of estimating a demand equation, using it to determine
the profit function and deriving the value function from that.

\item This would give you a functional form with micro foundations. If
individual utilities were derived from distance and other $x$'s, and we knew
the distribution of locations, this would connect the $\beta $ and $\theta $
to $\beta (d)$ and $\theta (d)$ and the characteristics of the adjacent
locations.

\item It would save on parameters, but would be harder to compute:

\begin{itemize}
\item we would need to aggregate up from location-level demand

\item we would have to think seriously about how to move from static profits
to value functions.
\end{itemize}

The fact that we don't do this seriously means that we should be primarily
thinking in terms of a reduced form value function.
\end{itemize}

%TCIMACRO{\TeXButton{EndFrame}{\end{frame}}}%
%BeginExpansion
\end{frame}%
%EndExpansion

%TCIMACRO{\TeXButton{BeginFrame}{\begin{frame}}}%
%BeginExpansion
\begin{frame}%
%EndExpansion

\frametitle{Seim (2006), cntd.}

\begin{itemize}
\item By going to the reduced form we introduce the \textquotedblleft too
many parameters\textquotedblright\ problem that we saw in demand models in
product space: the number of parameters grows geometrically in the number of
possible locations.

\item So she simplifies by saying the effect of a competitor is the same for
all locations that are equidistant from the firm.

\item The only difference between firms is $\eta _{i}^{a}$ which distributes
extreme value. (Could be an idiosyncratic costs of operating for different
potential firms?)

\item The only unobservable that is correlated across firms is $\xi ^{m}$
which is market-specific. Implicit assumption that we have very good data on
within market differences in profit shifters.
\end{itemize}

%TCIMACRO{\TeXButton{EndFrame}{\end{frame}}}%
%BeginExpansion
\end{frame}%
%EndExpansion

%TCIMACRO{\TeXButton{BeginFrame}{\begin{frame}}}%
%BeginExpansion
\begin{frame}%
%EndExpansion

\frametitle{Seim (2006):\ Methodology}

\begin{itemize}
\item The firm \textit{only knows} its \textit{own} unobservable. To form
expected profits for any location it has to form the expected number of
firms at all locations. These are expectations, rather than realizations,
because no firm knows the other firm's values for $\eta $.

\item The equilibrium expectation of the fraction that choose location $i$
is just the expected fraction of firms for whom the max profits come from
location $i.$

\item Since all firms are symmetric in observables, this has a simple form.

\item The true fraction of firms that will choose location $i$ is given by
the usual logit probabilities; i.e. 
\begin{equation*}
s_{i}=\frac{\exp (\rho \tilde{v}_{i}^{a})}{\sum_{k}\exp (\rho \tilde{v}%
_{k}^{a})}
\end{equation*}

where the $\rho $ comes from the use of a nested logit specification.
\end{itemize}

%TCIMACRO{\TeXButton{EndFrame}{\end{frame}}}%
%BeginExpansion
\end{frame}%
%EndExpansion

%TCIMACRO{\TeXButton{BeginFrame}{\begin{frame}}}%
%BeginExpansion
\begin{frame}%
%EndExpansion

\QTR{frametitle}{Seim (2006), cntd.}

\begin{itemize}
\item However, firms don't observe their rivals' $\eta $'s so have to form
an expectation:\ $p_{i}^{a,b}$= firm $a$'s conjecture of the probability
with which firm $b$ chooses location $i$:%
\begin{equation*}
p_{i}^{a,b}=\frac{\exp (\rho E_{\eta ^{-a}}\tilde{v}_{i}^{b})}{\sum_{k}\exp
(\rho E_{\eta ^{-a}}\tilde{v}_{k}^{b})}
\end{equation*}%
where $E_{\eta ^{-a}}$ means we have integrated out over the unobservable
firm specific probabilities and the implied number of entrants.

\item Now say that $a$'s perceived probability that any one of the other
entering firms will choose location $i$ is $p_{i}^{a}$ (the same for all $b$
since the value function depends on firm characteristics only through $\eta $%
), and that there are $\hat{N}$ firms that enter.

\item Then the total number of competitors, excluding itself, that firm $a$
expects to find in location $i$ is $E(n_{i}^{m})=\sum_{b\neq a}p_{i}^{a,b}=(%
\hat{N}-1)p_{i}^{a}$
\end{itemize}

%TCIMACRO{\TeXButton{EndFrame}{\end{frame}}}%
%BeginExpansion
\end{frame}%
%EndExpansion

%TCIMACRO{\TeXButton{BeginFrame}{\begin{frame}}}%
%BeginExpansion
\begin{frame}%
%EndExpansion

\QTR{frametitle}{Seim (2006), cntd.}

\begin{itemize}
\item Then firm $a$'s expected payoff in cell $i$ is: 
\begin{eqnarray*}
E_{\eta ^{-a}}v_{i}^{a} &=&\xi ^{m}+x_{m}^{i}\beta +[p_{i}^{a}(\hat{N}%
-1)+1]\theta +\sum_{j\in d1i,j}x_{m}^{j}\beta _{1}+\sum_{j\in
d2i,j}x_{m}^{j}\beta _{2}+ \\
&&\sum_{j\in d1i,j}p_{j}^{a}(\hat{N}-1)\theta _{1}+\sum_{j\in
d2i,j}p_{j}^{a}(\hat{N}-1)\theta _{2}+\eta _{i}^{a} \\
&=&E_{\eta ^{-a}}\tilde{v}_{i}^{a}+\eta _{i}^{a}
\end{eqnarray*}

\item Now assume a symmetric equilibrium so that $p_{i}^{a}=p_{i}^{b}=p_{i}.$
Substituting the above expression into the definition of $p_{i}^{a}$ gives
us 
\begin{equation*}
p_{i}=\frac{\exp (\rho E_{\eta ^{-a}}\tilde{v}_{i}^{a})}{\sum_{k}\exp (\rho
E_{\eta ^{-a}}\tilde{v}_{i}^{a})}
\end{equation*}

\item This gives us a fixed point for equilibrium conjectures on the $p_{i}$.

\item Note that this fixed point does not depend on any of the market
characteristics; that is choices among locations within the market depend
only on location-specific variables (e.g. $\xi _{m}$ cancel out).
\end{itemize}

%TCIMACRO{\TeXButton{EndFrame}{\end{frame}}}%
%BeginExpansion
\end{frame}%
%EndExpansion

%TCIMACRO{\TeXButton{BeginFrame}{\begin{frame}}}%
%BeginExpansion
\begin{frame}%
%EndExpansion

\QTR{frametitle}{Seim (2006), cntd.}

So there are 2 steps:

\begin{itemize}
\item Given $\hat{N}$, we can solve for equilibrium conjectures $p_{i}$ as a
function of the parameters of the model.

\begin{itemize}
\item There are $p_{i}$'s on both sides of the equation ($p_{i}=f(\hat{N}%
,x,\theta ,\beta ,p_{i})$; we iterate until the LHS\ equals the RHS.

\item This is done using the method of \textquotedblleft successive
approximations\textquotedblright . First pick values of $(\theta ,\beta )$.
Start with conjectures for $p_{i},$ calculate their implications and then
use those implications for the next iteration's conjectures. Repeat until we
converge on a value for $p_{i}$ for this $(\theta ,\beta ).$
\end{itemize}

\item Iterate over $(\theta ,\beta )$ to fit these equilibrium conjectures
to data on the shares of the entrants in each tract (M.L.E.). This provides
estimates of all but the market specific parameters. See the paper for
details on the latter.
\end{itemize}

%TCIMACRO{\TeXButton{EndFrame}{\end{frame}}}%
%BeginExpansion
\end{frame}%
%EndExpansion

%TCIMACRO{\TeXButton{BeginFrame}{\begin{frame}}}%
%BeginExpansion
\begin{frame}%
%EndExpansion

\QTR{frametitle}{Seim (2006):\ Notes}

\begin{itemize}
\item The model requires that $\hat{N}$ is not a function of $\eta _{-a}$ -
we need this to do the integral for expected values.

\begin{itemize}
\item One possible assumption:\ firms pay an initial entry cost before they
decide where they are going to enter and before they know their $\eta
_{a}^{i}$.
\end{itemize}

\item Another issue: after choices are made there is ex post regret; i.e.
firms will have made choices which ex post are not profitable.

\begin{itemize}
\item This would be OK\ in a dynamic framework

\item But it's odd in a 2-stage game where firms can't correct their
mistakes (e.g. through exit).
\end{itemize}

\item We have not considered existence or uniqueness. Existence follows from
continuity. There is no proof of uniqueness, and there will not be one -- at
least not without further special assumptions.
\end{itemize}

(Both of the last comments apply to all or many papers in this literature.)

%TCIMACRO{\TeXButton{EndFrame}{\end{frame}}}%
%BeginExpansion
\end{frame}%
%EndExpansion

%TCIMACRO{\TeXButton{BeginFrame}{\begin{frame}}}%
%BeginExpansion
\begin{frame}%
%EndExpansion

\QTR{frametitle}{Entry Models with Symmetric Information and Variable
Continuation Values.}

\begin{itemize}
\item Mazzeo (2003, RAND) and Toivanen and Waterson(1999) extend the simple
B \& R model to consider firm heterogeneity in continuation values keeping
fixed costs constant.

\item They assume symmetric information\ (unlike Seim).

\item Both assume a small number of entry locations which are fixed
exogeneously.

\item They then consider entry into all locations.

\item Again we work with a reduced form continuation value, but now it
varies with the locations of all entrants, and depends on the number of
entrants at each location.
\end{itemize}

%TCIMACRO{\TeXButton{EndFrame}{\end{frame}}}%
%BeginExpansion
\end{frame}%
%EndExpansion

%TCIMACRO{\TeXButton{BeginFrame}{\begin{frame}}}%
%BeginExpansion
\begin{frame}%
%EndExpansion

\QTR{frametitle}{Symmetric Information and Variable Continuation Values,
cntd.}

\begin{itemize}
\item Toivanen and Waterson deal with fast food outlets in Britain,
distinguishing between McDonalds and Burger King. \ Each chain decides how
many of its outlets should enter.

\item Mazzeo models independent motels choosing which interstate exits to
enter and with which quality level (high, medium or low).

\item Both assume that quality levels/types belong to a small discrete set
(with 2 or 3 elements) and these are the only differences among firms.

\item Continuation values for any firm choosing quality level $t=(1,...,T)$
in market $m$ are assumed to be 
\begin{equation*}
V_{tm}=x_{m}\beta _{t}+g(N_{1},..N_{T},\theta _{t})+\varepsilon _{tm.}
\end{equation*}
\end{itemize}

where $\bar{N}=N_{1},..N_{T}$ is the vector of the number of firms of each
type.

%TCIMACRO{\TeXButton{EndFrame}{\end{frame}}}%
%BeginExpansion
\end{frame}%
%EndExpansion

%TCIMACRO{\TeXButton{BeginFrame}{\begin{frame}}}%
%BeginExpansion
\begin{frame}%
%EndExpansion

\QTR{frametitle}{Symmetric Information and Variable Continuation Values,
cntd.}

\begin{itemize}
\item The parameters (specific to each type) are $\beta _{t}$ on the market
level variables and $\theta _{t}$ which parameterizes the effect of own-type
and other-type competition.

\item Both papers approximate $g$ with a set of dummy variables (since $\bar{%
N}$ takes on discrete values, so does $g(.,\theta _{t})$).

\item Note that the unobservables are constant across firms within quality
type, so that the model is symmetric conditional on type.

\item Unfortunately, even this simple type of heterogeneity in variable
profits leads to both possible nonexistence, and possible non-uniqueness
when equilibrium exists -- at least in a simultaneous move game.

\item Both authors resolve this by assuming that choices are sequential.
\end{itemize}

%TCIMACRO{\TeXButton{EndFrame}{\end{frame}}}%
%BeginExpansion
\end{frame}%
%EndExpansion

%TCIMACRO{\TeXButton{BeginFrame}{\begin{frame}}}%
%BeginExpansion
\begin{frame}%
%EndExpansion

\QTR{frametitle}{Toivanen and Waterson}

Assume Burger King moves first and the $\varepsilon $'s are draws from a
known distribution.

\begin{itemize}
\item Take a set of $ns$ draws of $(\varepsilon _{b},\varepsilon _{m})$.
Pick a $\theta $ to evaluate.

\item Let $EV^{j}(b,m)$ be the total entry value of Burger King (if $j=b$)
or McDonald's (if $j=m$) when there are $b$ Burger King outlets and $m$
McDonald's outlets.

\item Now solve the problem backwards, for BK moving first.

\begin{itemize}
\item If BK chooses $b$, predict McDonald's choice $m(b)$ to $%
max_{m}EV^{m}(b,m)$.

\item Given that response,\ BK chooses $b$ to $max_{b}EV^{b}(b,m(b))$.
\end{itemize}

\item This gives us $(b,m)$ for this $(\varepsilon ,\theta )$

\item Repeat for every draw of $\varepsilon $ to generate an expected $(\hat{%
b}(\theta ),\hat{m}(\theta ))$ given this $\theta $

\item Iterate over $\theta $ to minimize $E[((\hat{b}(\theta ),\hat{m}%
(\theta ))-(b,m))\ast f(x)]$ for exogenous entry determinants $x$.
\end{itemize}

%TCIMACRO{\TeXButton{EndFrame}{\end{frame}}}%
%BeginExpansion
\end{frame}%
%EndExpansion

%TCIMACRO{\TeXButton{BeginFrame}{\begin{frame}}}%
%BeginExpansion
\begin{frame}%
%EndExpansion

\QTR{frametitle}{Mazzeo}

We simplify by assuming there are only two types, $L$ and $H$.

\begin{itemize}
\item Let $EV^{l}(l,h)$ be the entry value of next type $l$ firm when
already $l$ low type and $h$ high type firms are active $(j=L,H)$.

\item Assume that firms play sequentially and make irrevocable decisions
about entry and product type before the next firm plays.

\item Firms anticipate subsequent entry in the usual way.

\item The last firm of each product type finds entry profitable and prefers
the chosen type to the alternatives.

\item Additional entry in either product type is not profitable.
\end{itemize}

%TCIMACRO{\TeXButton{EndFrame}{\end{frame}}}%
%BeginExpansion
\end{frame}%
%EndExpansion

%TCIMACRO{\TeXButton{BeginFrame}{\begin{frame}}}%
%BeginExpansion
\begin{frame}%
%EndExpansion

\QTR{frametitle}{Mazzeo, cntd.}

Therefore a Nash equilibrium is an ordered pair $(l,h)$ for which the
following inequalities hold 
\begin{equation*}
V_{l}(l-1,h)>0,\text{ }V_{l}(l,h)<0,\text{ }V_{l}(l-1,h)>V_{h}(l-1,h)
\end{equation*}%
and 
\begin{equation*}
V_{h}(l,h-1)>0,\text{ }V_{h}(l,h)<0,\text{ }V_{h}(l,h-1)>V_{l}(l,h-1)
\end{equation*}

\begin{itemize}
\item Mazzeo also considers an equilibrium where firms sink their sunk costs
without committing to whether they will be low or high quality.

\item Product types are selected among the entrants simultaneously in the
second stage.

\item He shows that the two equilibria can be different.

\item Mazzeo claims that both equilibria are unique, and he can simulate
them and match them to data.
\end{itemize}

%TCIMACRO{\TeXButton{EndFrame}{\end{frame}}}%
%BeginExpansion
\end{frame}%
%EndExpansion

%TCIMACRO{\TeXButton{BeginFrame}{\begin{frame}}}%
%BeginExpansion
\begin{frame}%
%EndExpansion

\QTR{frametitle}{Results of these Papers}

\begin{itemize}
\item Mazzeo finds strong returns to differentiation -- all entry by rivals
drives down profits but an incumbents profits fall by much more when the
rival is offering the same quality type as the incumbent.

\item Toivanen and Waterson find that McDonald's is much more likely to
enter a location where there are only Burger King outlets then where there
are McDonald's outlets.

\item I.e. consistent with Seim, and probably not surprisingly, both papers
find that differentiation matters for entry decisions, and hence probably
for profits and CS calculations.
\end{itemize}

%TCIMACRO{\TeXButton{EndFrame}{\end{frame}}}%
%BeginExpansion
\end{frame}%
%EndExpansion

%TCIMACRO{\TeXButton{BeginFrame}{\begin{frame}}}%
%BeginExpansion
\begin{frame}%
%EndExpansion

\QTR{frametitle}{General Issues in the Analysis of Entry Games.}

\begin{itemize}
\item \textit{We should be careful about doing either PS or CS\ calculations
without allowing for sufficient product differentiation. }

\item There is a real question of which equilibrium we pick out and why, and
this issue just gets more worrisome when we increase the number of types.

\begin{itemize}
\item There are a couple of ways around the non-uniqueness problem.

\item We could enumerate all possible equilibria and assign a probability to
each (perhaps depending on a vector of parameters), and estimate those
probabilities (or the extra parameter vector) along with the other
parameters (Tamer, 2003, RESTUD).

\item Or we could investigate whether the data can pick out the equilibrium
for us (this requires alternative assumptions).

\item Or we can get just bounds on probabilities (instead of probabilities
themselves), and use the bounds for estimation. See next class notes.
\end{itemize}
\end{itemize}

%TCIMACRO{\TeXButton{EndFrame}{\end{frame}}}%
%BeginExpansion
\end{frame}%
%EndExpansion

%TCIMACRO{\TeXButton{BeginFrame}{\begin{frame}}}%
%BeginExpansion
\begin{frame}%
%EndExpansion

\QTR{frametitle}{General Issues, cntd.}

\begin{itemize}
\item All reduced form value function models become harder as we increase
the number of types, because the number of parameters we estimate increases
at about the square of the number of types.

\begin{itemize}
\item One way around this is to define demand and cost functions directly on
the underlying \textquotedblleft characteristic\textquotedblright\ space,
just as we did in static analysis, and then use those profit functions
directly in the empirical analysis.

\item The inequalities estimator we will cover next class is well suited to
this approach.
\end{itemize}

\item Lack of True Dynamics.

\begin{itemize}
\item Introducing dynamics would solve the conceptual problems noted above

\item And it would use the information in the sequential nature of most data.

\item I.e., in many of these data sets we see when each firm entered, so we
could condition that firm's entry decision on firms existing in the market
at the time they enter.

\item The problems here are a mix of computational and conceptual, and they
will be covered in I.O.3 next semester.
\end{itemize}
\end{itemize}

%TCIMACRO{\TeXButton{EndFrame}{\end{frame}}}%
%BeginExpansion
\end{frame}%
%EndExpansion

\end{document}
