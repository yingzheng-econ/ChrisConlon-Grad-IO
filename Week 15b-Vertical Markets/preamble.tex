% Wrapper to use the canonical shared preamble
% Keep original class options for this folder if needed
\def\beamerclassoptions{[xcolor=pdftex,dvipsnames,table,mathserif,aspectratio=169]}
\documentclass[10pt,aspectratio=169]{beamer}

% silence some Metropolis warnings
\usepackage{silence}
\WarningFilter{beamerthememetropolis}{You need to compile with XeLaTeX or LuaLaTeX}
\WarningFilter{latexfont}{Font shape}
\WarningFilter{latexfont}{Some font}

% define custom colors
\usepackage{xcolor}
\definecolor{dark gray}{HTML}{444444}
\definecolor{light gray}{HTML}{777777}
\definecolor{dark red}{HTML}{BB0000}
\definecolor{dark green}{HTML}{00BB00}
\definecolor{RoyalBlue}{cmyk}{1, 0.50, 0, 0}

% configure metropolis
\usetheme[numbering=fraction]{metropolis}
\setbeamercolor{background canvas}{bg=white}
\setbeamercolor{frametitle}{bg=dark gray}
\setbeamercolor{alerted text}{fg=dark red}
\setbeamercolor{item projected}{bg=dark red}
\setbeamercolor{local structure}{fg=dark red}
\setbeamersize{text margin left=0.5cm,text margin right=0.5cm}
\setbeamercovered{transparent=10}

% use thicker lines
\makeatletter
\setlength{\metropolis@titleseparator@linewidth}{1pt}
\setlength{\metropolis@progressonsectionpage@linewidth}{1pt}
\makeatother

% custom bullet points
\setbeamertemplate{itemize item}{\color{dark red}$\blacktriangleright$}
\setbeamertemplate{itemize subitem}{\color{dark red}$\blacktriangleright$}
\setbeamertemplate{itemize subsubitem}{\color{dark red}$\blacktriangleright$}
\newcommand{\custombullet}{{\color{dark red}$\blacktriangleright$}\hspace{0.5em}}


% imports
\usepackage[english]{babel}
\usepackage[utf8]{inputenc}
\usepackage{amsthm}
\usepackage{amssymb}
\usepackage{amsmath}
\usepackage{amsfonts}
\usepackage{mathtools}
\usepackage{mathabx}
\usepackage{stmaryrd}
\usepackage{graphicx}
\usepackage{hyperref}
\usepackage{xfrac}
\usepackage{appendixnumberbeamer}
\usepackage{tabularx}

% check and x marks
\usepackage{pifont}
\newcommand{\cmark}{{\color{dark green}\ding{51}}\hspace{0.3em}}
\newcommand{\xmark}{{\color{dark red}\ding{55}}\hspace{0.5em}}


% use classic font for math
\usepackage[T1]{fontenc} % Needed for Type1 Concrete \usepackage{concmath}
\usefonttheme{serif}
\usefonttheme{professionalfonts}
\usepackage{concmath}
\setbeamerfont{equation}{size=\tiny}



% diagrams
\usepackage{tikz}
\usetikzlibrary{decorations.pathreplacing}

% references
\usepackage[natbibapa]{apacite}
\bibliographystyle{apacite}
\renewcommand{\bibsection}{}

% use ampersands instead of "and" for text citations
\AtBeginDocument{\renewcommand{\BBAB}{\&}}

% possessive cites
\makeatletter
\patchcmd{\NAT@test}{\else \NAT@nm}{\else \NAT@nmfmt{\NAT@nm}}{}{}
\DeclareRobustCommand\citepos
  {\begingroup
   \let\NAT@nmfmt\NAT@posfmt
   \NAT@swafalse\let\NAT@ctype\z@\NAT@partrue
   \@ifstar{\NAT@fulltrue\NAT@citetp}{\NAT@fullfalse\NAT@citetp}}
\let\NAT@orig@nmfmt\NAT@nmfmt
\def\NAT@posfmt#1{\NAT@orig@nmfmt{#1's}}
\makeatother

% spaced-out lists
\newenvironment{wideitemize}{\itemize\addtolength{\itemsep}{10pt}}{\enditemize}
\newenvironment{wideenumerate}{\enumerate\addtolength{\itemsep}{10pt}}{\endenumerate}

% replace footnotes with buttons
\usepackage[absolute,overlay]{textpos}
\newcounter{beamerpausessave}
\newcommand{\always}[1]{
    \setcounter{beamerpausessave}{\value{beamerpauses}}
    \setcounter{beamerpauses}{0}
    \pause
    #1 
    \setcounter{beamerpauses}{\value{beamerpausessave}}
    \addtocounter{beamerpauses}{-1}
    \pause
}
\newcommand{\buttons}[1]{\always{
    \begin{textblock*}{\paperwidth}(0.015\textwidth, 1.022\textheight)
        \scriptsize
        #1
    \end{textblock*}
}}
\newcommand{\appendixbuttons}[1]{\always{
    \begin{textblock*}{\paperwidth}(0.015\textwidth, 1.043\textheight)
        \scriptsize
        #1
    \end{textblock*}
}}
\newcommand{\goto}[2]{\hyperlink{#1}{{\color{dark red}$\smalltriangleright$} #2}\hspace{0.5em}}
\newcommand{\goback}[2]{\hyperlink{#1}{{\color{dark red}$\smalltriangleleft$} #2}\hspace{0.5em}}

% custom appendix
\renewcommand{\appendixname}{\texorpdfstring{\translate{Appendix}}{Appendix}}

% change color of cites and URLs
\let\oldcite\cite
\let\oldcitet\citet
\let\oldcitep\citep
\let\oldcitepos\citepos
\let\oldcitetalias\citetalias
\let\oldcitepalias\citepalias
\let\oldurl\url
\def\cite#1#{\citeaux{#1}}
\def\citet#1#{\citetaux{#1}}
\def\citep#1#{\citepaux{#1}}
\def\citepos#1#{\citeposaux{#1}}
\def\citetalias#1#{\citetaliasaux{#1}}
\def\citepalias#1#{\citepaliasaux{#1}}
\def\url#1#{\urlaux{#1}}
\newcommand*\citeaux[2]{{\color{light gray}\oldcite#1{#2}}}
\newcommand*\citetaux[2]{{\color{light gray}\oldcitet#1{#2}}}
\newcommand*\citepaux[2]{{\color{light gray}\oldcitep#1{#2}}}
\newcommand*\urlaux[2]{{\color{light gray}\oldurl#1{#2}}}
\newcommand*\citeposaux[2]{{\color{light gray}\oldcitepos#1{#2}}}
\newcommand*\citetaliasaux[2]{{\color{light gray}\oldcitetalias#1{#2}}}
\newcommand*\citepaliasaux[2]{{\color{light gray}\oldcitepalias#1{#2}}}

% custom math commands
\DeclareMathOperator*{\argmax}{argmax}
\DeclareMathOperator*{\argmin}{argmin}
\renewcommand{\Pr}{\mathbb{P}}
\newcommand{\E}{\mathbb{E}}
\newcommand{\Var}{\mathbb{V}}
\newcommand{\Cov}{\mathbb{C}}
\newcommand{\overbar}[1]{\mkern 1.5mu\overline{\mkern-1.5mu#1\mkern-1.5mu}\mkern 1.5mu}
\newcommand{\abs}[1]{\lvert#1\rvert}
\newcommand{\norm}[1]{\lVert#1\rVert}

% tables
\usepackage{booktabs}
\usepackage{colortbl}
\usepackage{multirow}
\usepackage{makecell}
\arrayrulecolor{dark red}

% custom date
\usepackage{datetime}
\newdateformat{monthyeardate}{\monthname[\THEMONTH] \THEYEAR}

% fix pauses with graphics
\usepackage{../resources/fixpauseincludegraphics}


% packages
%\usepackage{newpxmath} % math font is Palatino compatible
%\usepackage[nomath]{fontspec}

\usepackage{setspace}
\usepackage{xcolor}
\usepackage{soul} % for \st
\usepackage{hyperref} % for links
\definecolor{links}{HTML}{2A1B81}
\hypersetup{colorlinks,linkcolor=,urlcolor=links}


% table stuff
\usepackage{chronosys}
\usepackage{verbatim}
% \pagenumbering{arabic}
\usepackage{tabularx}
\usepackage{booktabs}
\usepackage{ragged2e}
\usepackage{mathtools}

% R Code
\usepackage{listings}
\usepackage{courier}
\lstset{basicstyle=\scriptsize\ttfamily,breaklines=true}
\lstset{framextopmargin=50pt,frame=bottomline}

% themes
\usetheme[progressbar=frametitle, block=fill]{metropolis}
\useoutertheme{metropolis}
\useinnertheme{metropolis}

% colors
\definecolor{dimwhite}{rgb}{0.99, 0.99, 0.99}
\definecolor{charcoal}{rgb}{0.21, 0.27, 0.31}
\definecolor{slategray}{rgb}{0.44, 0.5, 0.56}
\definecolor{dimgray}{rgb}{0.41, 0.41, 0.41}
\definecolor{bleudefrance}{rgb}{0.19, 0.55, 0.91}

% beamer options
\setbeamercolor{author}{fg=charcoal}
\setbeamercolor{background canvas}{bg=white}
\setbeamercolor{section in toc}{fg=charcoal}
\setbeamercolor{subsection in toc}{fg=dimgray}
\setbeamercolor{frametitle}{bg=dimwhite, fg=charcoal}
\setbeamercolor{progress bar}{fg=slategray, bg=fg!50!black!30}
\setbeamercovered{transparent}
\setbeamertemplate{itemize items}[triangle]
\setbeamertemplate{itemize subitem}[circle]
\setbeamertemplate{itemize subsubitem}[square]
\setbeamersize{text margin left=7mm,text margin right=7mm} 

% new commands
\newcommand{\q}[1]{``#1''}
\newcommand{\hs}[1]{\textsc{\hfill\scriptsize\color{dimgray}#1}}
\newcommand{\g}[1]{{\color{gray}#1}}
\newcommand{\dg}[1]{{\color{dimgray}#1}}
\newcommand{\sg}[1]{{\color{slategray}#1}}
\newcommand{\bdf}[1]{{\color{bleudefrance}#1}}
\newcommand{\itemcolor}[1]{\renewcommand{\makelabel}[1]{\color{#1}\hfil ##1}}
\newcommand\Wider[2][2em]{
\makebox[\linewidth][c]{
  \begin{minipage}{\dimexpr\textwidth+#1\relax}
  \raggedright#2
  \end{minipage}
  }
}

% misc
\linespread{1.35}

% Math stuff
\newcommand{\norm}[1]{\left\lVert#1\right\rVert}
\newcommand{\R}{\mathbb{R}}
\newcommand{\E}{\mathbb{E}}
\newcommand{\V}{\mathbb{V}}
\newcommand{\probP}{\mathbb{P}}
\newcommand{\ol}{\overline}
%\newcommand{\ul}{\underline}
\newcommand{\pp}{{\prime \prime}}
\newcommand{\ppp}{{\prime \prime \prime}}
\newcommand{\policy}{\gamma}
\newcommand{\plim}{ \overset{p}{\to}}
\newcommand{\hnot}{ \overset{H_0}{\to}}

% Causal Graphs
\usetikzlibrary{shapes,decorations,arrows,calc,arrows.meta,fit,positioning}
\tikzset{
    -Latex,auto,node distance =1 cm and 1 cm,semithick,
    state/.style ={ellipse, draw, minimum width = 0.7 cm},
    point/.style = {circle, draw, inner sep=0.04cm,fill,node contents={}},
    bidirected/.style={Latex-Latex,dashed},
    el/.style = {inner sep=2pt, align=left, sloped}
}