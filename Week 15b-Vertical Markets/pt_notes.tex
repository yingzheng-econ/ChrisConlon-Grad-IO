% Wrapper to use the canonical shared preamble
\documentclass[10pt,aspectratio=169]{beamer}

% silence some Metropolis warnings
\usepackage{silence}
\WarningFilter{beamerthememetropolis}{You need to compile with XeLaTeX or LuaLaTeX}
\WarningFilter{latexfont}{Font shape}
\WarningFilter{latexfont}{Some font}

% define custom colors
\usepackage{xcolor}
\definecolor{dark gray}{HTML}{444444}
\definecolor{light gray}{HTML}{777777}
\definecolor{dark red}{HTML}{BB0000}
\definecolor{dark green}{HTML}{00BB00}
\definecolor{RoyalBlue}{cmyk}{1, 0.50, 0, 0}

% configure metropolis
\usetheme[numbering=fraction]{metropolis}
\setbeamercolor{background canvas}{bg=white}
\setbeamercolor{frametitle}{bg=dark gray}
\setbeamercolor{alerted text}{fg=dark red}
\setbeamercolor{item projected}{bg=dark red}
\setbeamercolor{local structure}{fg=dark red}
\setbeamersize{text margin left=0.5cm,text margin right=0.5cm}
\setbeamercovered{transparent=10}

% use thicker lines
\makeatletter
\setlength{\metropolis@titleseparator@linewidth}{1pt}
\setlength{\metropolis@progressonsectionpage@linewidth}{1pt}
\makeatother

% custom bullet points
\setbeamertemplate{itemize item}{\color{dark red}$\blacktriangleright$}
\setbeamertemplate{itemize subitem}{\color{dark red}$\blacktriangleright$}
\setbeamertemplate{itemize subsubitem}{\color{dark red}$\blacktriangleright$}
\newcommand{\custombullet}{{\color{dark red}$\blacktriangleright$}\hspace{0.5em}}


% imports
\usepackage[english]{babel}
\usepackage[utf8]{inputenc}
\usepackage{amsthm}
\usepackage{amssymb}
\usepackage{amsmath}
\usepackage{amsfonts}
\usepackage{mathtools}
\usepackage{mathabx}
\usepackage{stmaryrd}
\usepackage{graphicx}
\usepackage{hyperref}
\usepackage{xfrac}
\usepackage{appendixnumberbeamer}
\usepackage{tabularx}

% check and x marks
\usepackage{pifont}
\newcommand{\cmark}{{\color{dark green}\ding{51}}\hspace{0.3em}}
\newcommand{\xmark}{{\color{dark red}\ding{55}}\hspace{0.5em}}


% use classic font for math
\usepackage[T1]{fontenc} % Needed for Type1 Concrete \usepackage{concmath}
\usefonttheme{serif}
\usefonttheme{professionalfonts}
\usepackage{concmath}
\setbeamerfont{equation}{size=\tiny}



% diagrams
\usepackage{tikz}
\usetikzlibrary{decorations.pathreplacing}

% references
\usepackage[natbibapa]{apacite}
\bibliographystyle{apacite}
\renewcommand{\bibsection}{}

% use ampersands instead of "and" for text citations
\AtBeginDocument{\renewcommand{\BBAB}{\&}}

% possessive cites
\makeatletter
\patchcmd{\NAT@test}{\else \NAT@nm}{\else \NAT@nmfmt{\NAT@nm}}{}{}
\DeclareRobustCommand\citepos
  {\begingroup
   \let\NAT@nmfmt\NAT@posfmt
   \NAT@swafalse\let\NAT@ctype\z@\NAT@partrue
   \@ifstar{\NAT@fulltrue\NAT@citetp}{\NAT@fullfalse\NAT@citetp}}
\let\NAT@orig@nmfmt\NAT@nmfmt
\def\NAT@posfmt#1{\NAT@orig@nmfmt{#1's}}
\makeatother

% spaced-out lists
\newenvironment{wideitemize}{\itemize\addtolength{\itemsep}{10pt}}{\enditemize}
\newenvironment{wideenumerate}{\enumerate\addtolength{\itemsep}{10pt}}{\endenumerate}

% replace footnotes with buttons
\usepackage[absolute,overlay]{textpos}
\newcounter{beamerpausessave}
\newcommand{\always}[1]{
    \setcounter{beamerpausessave}{\value{beamerpauses}}
    \setcounter{beamerpauses}{0}
    \pause
    #1 
    \setcounter{beamerpauses}{\value{beamerpausessave}}
    \addtocounter{beamerpauses}{-1}
    \pause
}
\newcommand{\buttons}[1]{\always{
    \begin{textblock*}{\paperwidth}(0.015\textwidth, 1.022\textheight)
        \scriptsize
        #1
    \end{textblock*}
}}
\newcommand{\appendixbuttons}[1]{\always{
    \begin{textblock*}{\paperwidth}(0.015\textwidth, 1.043\textheight)
        \scriptsize
        #1
    \end{textblock*}
}}
\newcommand{\goto}[2]{\hyperlink{#1}{{\color{dark red}$\smalltriangleright$} #2}\hspace{0.5em}}
\newcommand{\goback}[2]{\hyperlink{#1}{{\color{dark red}$\smalltriangleleft$} #2}\hspace{0.5em}}

% custom appendix
\renewcommand{\appendixname}{\texorpdfstring{\translate{Appendix}}{Appendix}}

% change color of cites and URLs
\let\oldcite\cite
\let\oldcitet\citet
\let\oldcitep\citep
\let\oldcitepos\citepos
\let\oldcitetalias\citetalias
\let\oldcitepalias\citepalias
\let\oldurl\url
\def\cite#1#{\citeaux{#1}}
\def\citet#1#{\citetaux{#1}}
\def\citep#1#{\citepaux{#1}}
\def\citepos#1#{\citeposaux{#1}}
\def\citetalias#1#{\citetaliasaux{#1}}
\def\citepalias#1#{\citepaliasaux{#1}}
\def\url#1#{\urlaux{#1}}
\newcommand*\citeaux[2]{{\color{light gray}\oldcite#1{#2}}}
\newcommand*\citetaux[2]{{\color{light gray}\oldcitet#1{#2}}}
\newcommand*\citepaux[2]{{\color{light gray}\oldcitep#1{#2}}}
\newcommand*\urlaux[2]{{\color{light gray}\oldurl#1{#2}}}
\newcommand*\citeposaux[2]{{\color{light gray}\oldcitepos#1{#2}}}
\newcommand*\citetaliasaux[2]{{\color{light gray}\oldcitetalias#1{#2}}}
\newcommand*\citepaliasaux[2]{{\color{light gray}\oldcitepalias#1{#2}}}

% custom math commands
\DeclareMathOperator*{\argmax}{argmax}
\DeclareMathOperator*{\argmin}{argmin}
\renewcommand{\Pr}{\mathbb{P}}
\newcommand{\E}{\mathbb{E}}
\newcommand{\Var}{\mathbb{V}}
\newcommand{\Cov}{\mathbb{C}}
\newcommand{\overbar}[1]{\mkern 1.5mu\overline{\mkern-1.5mu#1\mkern-1.5mu}\mkern 1.5mu}
\newcommand{\abs}[1]{\lvert#1\rvert}
\newcommand{\norm}[1]{\lVert#1\rVert}

% tables
\usepackage{booktabs}
\usepackage{colortbl}
\usepackage{multirow}
\usepackage{makecell}
\arrayrulecolor{dark red}

% custom date
\usepackage{datetime}
\newdateformat{monthyeardate}{\monthname[\THEMONTH] \THEYEAR}

% fix pauses with graphics
\usepackage{../resources/fixpauseincludegraphics}


% silence some Metropolis warnings
\usepackage{silence}
\WarningFilter{beamerthememetropolis}{You need to compile with XeLaTeX or LuaLaTeX}
\WarningFilter{latexfont}{Font shape}
\WarningFilter{latexfont}{Some font}

% define custom colors
\definecolor{dark gray}{HTML}{444444}
\definecolor{light gray}{HTML}{777777}
\definecolor{dark red}{HTML}{BB0000}
\definecolor{dark green}{HTML}{00BB00}

% configure metropolis
\usetheme[numbering=fraction]{metropolis}
\setbeamercolor{background canvas}{bg=white}
\setbeamercolor{frametitle}{bg=dark gray}
\setbeamercolor{alerted text}{fg=dark red}
\setbeamercolor{item projected}{bg=dark red}
\setbeamercolor{local structure}{fg=dark red}
\setbeamersize{text margin left=0.5cm,text margin right=0.5cm}
\setbeamercovered{transparent=10}

% use thicker lines
\makeatletter
\setlength{\metropolis@titleseparator@linewidth}{1pt}
\setlength{\metropolis@progressonsectionpage@linewidth}{1pt}
\makeatother

% custom bullet points

% Minimal wrapper to use the canonical shared preamble for Passthrough lectures.
\def\beamerclassoptions{[10pt,aspectratio=169]}
\documentclass[10pt,aspectratio=169]{beamer}

% silence some Metropolis warnings
\usepackage{silence}
\WarningFilter{beamerthememetropolis}{You need to compile with XeLaTeX or LuaLaTeX}
\WarningFilter{latexfont}{Font shape}
\WarningFilter{latexfont}{Some font}

% define custom colors
\usepackage{xcolor}
\definecolor{dark gray}{HTML}{444444}
\definecolor{light gray}{HTML}{777777}
\definecolor{dark red}{HTML}{BB0000}
\definecolor{dark green}{HTML}{00BB00}
\definecolor{RoyalBlue}{cmyk}{1, 0.50, 0, 0}

% configure metropolis
\usetheme[numbering=fraction]{metropolis}
\setbeamercolor{background canvas}{bg=white}
\setbeamercolor{frametitle}{bg=dark gray}
\setbeamercolor{alerted text}{fg=dark red}
\setbeamercolor{item projected}{bg=dark red}
\setbeamercolor{local structure}{fg=dark red}
\setbeamersize{text margin left=0.5cm,text margin right=0.5cm}
\setbeamercovered{transparent=10}

% use thicker lines
\makeatletter
\setlength{\metropolis@titleseparator@linewidth}{1pt}
\setlength{\metropolis@progressonsectionpage@linewidth}{1pt}
\makeatother

% custom bullet points
\setbeamertemplate{itemize item}{\color{dark red}$\blacktriangleright$}
\setbeamertemplate{itemize subitem}{\color{dark red}$\blacktriangleright$}
\setbeamertemplate{itemize subsubitem}{\color{dark red}$\blacktriangleright$}
\newcommand{\custombullet}{{\color{dark red}$\blacktriangleright$}\hspace{0.5em}}


% imports
\usepackage[english]{babel}
\usepackage[utf8]{inputenc}
\usepackage{amsthm}
\usepackage{amssymb}
\usepackage{amsmath}
\usepackage{amsfonts}
\usepackage{mathtools}
\usepackage{mathabx}
\usepackage{stmaryrd}
\usepackage{graphicx}
\usepackage{hyperref}
\usepackage{xfrac}
\usepackage{appendixnumberbeamer}
\usepackage{tabularx}

% check and x marks
\usepackage{pifont}
\newcommand{\cmark}{{\color{dark green}\ding{51}}\hspace{0.3em}}
\newcommand{\xmark}{{\color{dark red}\ding{55}}\hspace{0.5em}}


% use classic font for math
\usepackage[T1]{fontenc} % Needed for Type1 Concrete \usepackage{concmath}
\usefonttheme{serif}
\usefonttheme{professionalfonts}
\usepackage{concmath}
\setbeamerfont{equation}{size=\tiny}



% diagrams
\usepackage{tikz}
\usetikzlibrary{decorations.pathreplacing}

% references
\usepackage[natbibapa]{apacite}
\bibliographystyle{apacite}
\renewcommand{\bibsection}{}

% use ampersands instead of "and" for text citations
\AtBeginDocument{\renewcommand{\BBAB}{\&}}

% possessive cites
\makeatletter
\patchcmd{\NAT@test}{\else \NAT@nm}{\else \NAT@nmfmt{\NAT@nm}}{}{}
\DeclareRobustCommand\citepos
  {\begingroup
   \let\NAT@nmfmt\NAT@posfmt
   \NAT@swafalse\let\NAT@ctype\z@\NAT@partrue
   \@ifstar{\NAT@fulltrue\NAT@citetp}{\NAT@fullfalse\NAT@citetp}}
\let\NAT@orig@nmfmt\NAT@nmfmt
\def\NAT@posfmt#1{\NAT@orig@nmfmt{#1's}}
\makeatother

% spaced-out lists
\newenvironment{wideitemize}{\itemize\addtolength{\itemsep}{10pt}}{\enditemize}
\newenvironment{wideenumerate}{\enumerate\addtolength{\itemsep}{10pt}}{\endenumerate}

% replace footnotes with buttons
\usepackage[absolute,overlay]{textpos}
\newcounter{beamerpausessave}
\newcommand{\always}[1]{
    \setcounter{beamerpausessave}{\value{beamerpauses}}
    \setcounter{beamerpauses}{0}
    \pause
    #1 
    \setcounter{beamerpauses}{\value{beamerpausessave}}
    \addtocounter{beamerpauses}{-1}
    \pause
}
\newcommand{\buttons}[1]{\always{
    \begin{textblock*}{\paperwidth}(0.015\textwidth, 1.022\textheight)
        \scriptsize
        #1
    \end{textblock*}
}}
\newcommand{\appendixbuttons}[1]{\always{
    \begin{textblock*}{\paperwidth}(0.015\textwidth, 1.043\textheight)
        \scriptsize
        #1
    \end{textblock*}
}}
\newcommand{\goto}[2]{\hyperlink{#1}{{\color{dark red}$\smalltriangleright$} #2}\hspace{0.5em}}
\newcommand{\goback}[2]{\hyperlink{#1}{{\color{dark red}$\smalltriangleleft$} #2}\hspace{0.5em}}

% custom appendix
\renewcommand{\appendixname}{\texorpdfstring{\translate{Appendix}}{Appendix}}

% change color of cites and URLs
\let\oldcite\cite
\let\oldcitet\citet
\let\oldcitep\citep
\let\oldcitepos\citepos
\let\oldcitetalias\citetalias
\let\oldcitepalias\citepalias
\let\oldurl\url
\def\cite#1#{\citeaux{#1}}
\def\citet#1#{\citetaux{#1}}
\def\citep#1#{\citepaux{#1}}
\def\citepos#1#{\citeposaux{#1}}
\def\citetalias#1#{\citetaliasaux{#1}}
\def\citepalias#1#{\citepaliasaux{#1}}
\def\url#1#{\urlaux{#1}}
\newcommand*\citeaux[2]{{\color{light gray}\oldcite#1{#2}}}
\newcommand*\citetaux[2]{{\color{light gray}\oldcitet#1{#2}}}
\newcommand*\citepaux[2]{{\color{light gray}\oldcitep#1{#2}}}
\newcommand*\urlaux[2]{{\color{light gray}\oldurl#1{#2}}}
\newcommand*\citeposaux[2]{{\color{light gray}\oldcitepos#1{#2}}}
\newcommand*\citetaliasaux[2]{{\color{light gray}\oldcitetalias#1{#2}}}
\newcommand*\citepaliasaux[2]{{\color{light gray}\oldcitepalias#1{#2}}}

% custom math commands
\DeclareMathOperator*{\argmax}{argmax}
\DeclareMathOperator*{\argmin}{argmin}
\renewcommand{\Pr}{\mathbb{P}}
\newcommand{\E}{\mathbb{E}}
\newcommand{\Var}{\mathbb{V}}
\newcommand{\Cov}{\mathbb{C}}
\newcommand{\overbar}[1]{\mkern 1.5mu\overline{\mkern-1.5mu#1\mkern-1.5mu}\mkern 1.5mu}
\newcommand{\abs}[1]{\lvert#1\rvert}
\newcommand{\norm}[1]{\lVert#1\rVert}

% tables
\usepackage{booktabs}
\usepackage{colortbl}
\usepackage{multirow}
\usepackage{makecell}
\arrayrulecolor{dark red}

% custom date
\usepackage{datetime}
\newdateformat{monthyeardate}{\monthname[\THEMONTH] \THEYEAR}

% fix pauses with graphics
\usepackage{../resources/fixpauseincludegraphics}




\title []{Pass-through}
\author{C.Conlon }
\institute{Internal Notes }
\date{Fall 2022}
\setbeamerfont{equation}{size=\tiny}
\begin{document}

\begin{frame}
\titlepage
\end{frame}




%%%%%%%%%%%%%%%%%%%%%%%%%%%%%%%%%%%%%%%%%%%%%%%%%%
%%%%%%%%%%%%%%%%%%%%%%%%%%%%%%%%%%%%%%%%%%%%%%%%%%%


\begin{frame}{Villas Boas (ReStud 2007)}
Retailer and Wholesaler FOC given by:
\begin{align*}
\symbf{p^r} &= \underbrace{\symbf{p^w} +\symbf{c^r}}_{\symbf{mc^r}} -(\mathcal{H}_r \odot \Delta_{r}(\symbf{p^r}))^{-1} \symbf{s}(\symbf{p^r})\\
\symbf{p^w}  &= \symbf{mc^w} + \left(\mathcal{H}_{w} \odot \left( \frac{\partial \symbf{p^r}}{\partial \symbf{p^w}} \cdot  \Delta_r(\symbf{p^r} ) \right) \right)^{-1} \symbf{s}(\symbf{p^r})
\end{align*}
\begin{itemize}
  \item $\Delta_r$ is matrix of (retail) demand derivatives $\frac{\partial\, \symbf{s}}{\partial\, \symbf{p}}$.
\item $\mathcal{H}_r,\mathcal{H}_w$  ownership matrix $(j,k)=1$ if both products sold by same retailer/wholesaler.
\item $\frac{\partial\, \symbf{p^r}}{\partial\, \symbf{p^w}}$ is the \alert{pass-through matrix} (NEW!)
\end{itemize}
Challenge: We want $\symbf{p^r}(\symbf{p^w})$ and $\symbf{mc^w}$ but we only have implicit solution for retailer FOC.
\end{frame}

\begin{frame}{How do we get pass-through?}
The \alert{pass-through matrix} $\frac{\partial \symbf{p^r}}{\partial \symbf{p^w}}$ can be obtained in one of two ways:
\begin{enumerate}
\item Numerically: perturbing the retailer's marginal costs for each possible choice of $k$ and solving
\begin{align*}
\symbf{p^r} &=\symbf{mc^r} + e_k -(\mathcal{H}_r \odot \Delta_{r}(\symbf{p^r}))^{-1} \symbf{s}(\symbf{p^r})\\
\end{align*}
(Use Morrow Skerlos (2011) formulation and solve for every $(j,k)$ pair).
\item Analytic: Use the retailer's FOC and apply the implicit function theorem.
\begin{align}
\tag{retailer FOC}
 f(\symbf{p^r},\symbf{mc^r}) &\equiv \symbf{p^r}  - \symbf{mc^r}-  \left(\mathcal{H}_{r} \odot \Delta(\symbf{p^r}) \right)^{-1} \symbf{s}(\symbf{p^r})=0 
\end{align}
See Jaffe Weyl (AEJM 2013) or Miller Weinberg (2017 Appendix E) or Conlon Rao (2022).\\
\alert{This is what PyBLP does}.
  \end{enumerate}

\end{frame}

\begin{frame}{Multivariate IFT: Easy Part}
The multivariate IFT says that for some system of $J$ nonlinear equations 
\begin{align*}
f(\symbf{p^r},\symbf{p^w}) \equiv [F_1(\symbf{p^r},\symbf{p^w}), \ldots, F_J(\symbf{p^r},\symbf{p^w})]=[0,\ldots,0]
\end{align*}
with $J$ endogenous variables $\symbf{p^r}$ and $J$ exogenous parameters $\symbf{p^w}$.
\begin{align}
\label{eq:ptr_matrix}
\tag{PTR}
\frac{\partial \symbf{p^r}}{\partial \symbf{p^w}}
=-\left(\begin{array}{ccc}
\frac{\partial F_{1}}{\partial p_{1}^r} & \ldots & \frac{\partial F_{1}}{\partial p_{J}^r} \\
\ldots & \ldots & \ldots \\
\frac{\partial F_{J}}{\partial p_{1}^r} & \ldots & \frac{\partial F_{J}}{\partial p_{J}^r}
\end{array}\right)^{-1} \cdot \underbrace{\left(\begin{array}{l}
\frac{\partial F_{1}}{\partial p_{k}^w} \\
\ldots \\
\frac{\partial F_{J}}{\partial p_{k}^w}
\end{array}\right)}_{= -\mathbb{I}_J}
\end{align}
Because the system of equations is additive in $\symbf{mc^r} = \symbf{c^r} + \symbf{p^w}$ this simplifies dramatically.
\end{frame}


\begin{frame}{Multivariate IFT: Hard Part}
Use the substitution $\Omega(\symbf{p^r}) \equiv \mathcal{H}_r \odot \Delta_{r}(\symbf{p^r})$, and differentiate the wholesalers' system of FOC's with respect to $p_l$, to get the $J \times J$ matrix with columns $l$ given by:
\begin{align}
\frac{\partial f(\symbf{p^r},\symbf{p^w})}{\partial p_l^r} \equiv e_l - \Omega^{-1}(\symbf{p^r})
\left[  \mathcal{H}_{r} \odot \frac{\partial\, \Delta(\symbf{p^r})}{\partial\, p_l^r} \right]
\Omega^{-1}(\symbf{p^r})\,
\symbf{s}(\symbf{p^r}) -\Omega^{-1}(\symbf{p^r})\, \frac{\partial \symbf{s}(\symbf{p^r})}{\partial p_l^r}.
\end{align}
The complicated piece is the demand Hessian: a $J \times J \times J$ tensor with elements $(j,k,l)$, $\frac{\partial^2 s_j}{\partial p_k^r \partial p_l^r} = \frac{\partial^2 \symbf{s}}{\partial \symbf{p^r} \partial p_l^r}=\frac{\partial\, \Delta(\symbf{p^r})}{\partial\, p_l^r}$.\\

This also shows a key relationship between \alert{pass through} and \alert{demand curvature} (2nd derivatives).
\end{frame}



\begin{frame}{Pass-through Counterfactuals?}
\footnotesize
How do we solve for $p^w$ under a counterfactual pass-through matrix?
\begin{itemize}
\item Idea: pass-through only augments the matrix $\Delta_r(\symbf{p^r})$.
\item Example: a constant sales tax rate $P \equiv \frac{\partial \symbf{p^r}}{\partial \symbf{p^w}} = \text{diag}(1+\tau_r)$
\end{itemize}
\begin{align*}
\symbf{p^w}  &= \symbf{mc^w} + \left(\mathcal{H}_{w} \odot \left( \frac{\partial \symbf{p^r}}{\partial \symbf{p^w}} \cdot  \Delta_r(\symbf{p^r} ) \right) \right)^{-1} \symbf{s}(\symbf{p^r})
\end{align*}

Adapt the Morrow Skerlos $\zeta$ fixed point where $P \Delta(\symbf{p}_t) = P \Lambda_t\left(\symbf{p}_t\right)- P \Gamma_t\left(\symbf{p}_t\right)$
\begin{align*}
\symbf{p}_t &\leftrightarrow \symbf{c}_t+\symbf{\zeta}_t\left(\symbf{p}_t\right) \quad \text { where }\\
 \symbf{\zeta}_t\left(\symbf{p}_t\right)&=\Lambda_t\left(\symbf{p}_t\right)^{-1} \alert{P^{-1}}\left[\mathcal{H}_t^* \odot \alert{P}\, \Gamma_t\left(\symbf{p}_t\right)\right]\left(\symbf{p}_t-\symbf{c}_t\right)-\Lambda_t\left(\symbf{p}_t\right)^{-1} \alert{P^{-1}} \symbf{s}_t\left(\symbf{p}_t\right)
\end{align*}
For diagonal $P$ (not sure about general case with Hadamard product):
\begin{align*}
 \symbf{\zeta}_t\left(\symbf{p}_t\right)=\Lambda_t\left(\symbf{p}_t\right)^{-1}\left[\mathcal{H}_t^* \odot  \Gamma_t\left(\symbf{p}_t\right)\right]\left(\symbf{p}_t-\symbf{c}_t\right)-\Lambda_t\left(\symbf{p}_t\right)^{-1} \alert{P^{-1}} \symbf{s}_t\left(\symbf{p}_t\right)
\end{align*}
 
\end{frame}


\end{document}